\documentclass[11pt]{article}
\usepackage[dvipsnames]{xcolor}
\usepackage{latexsym}
\usepackage{amsmath}
\usepackage{mathrsfs}
\usepackage{tikz-cd}
\usepackage{amssymb}
\usepackage{amsthm}
\usepackage{epsfig}
\usepackage{graphicx}
\usepackage{float}
\newcommand{\handout}[5]{
	\noindent
	\begin{center}
		\framebox{
			\vbox{
				\hbox to 5.78in { {\bf Hyperbolicity Course Notes} \hfill #2 }
				\vspace{4mm}
				\hbox to 5.78in { {\Large \hfill #5  \hfill} }
				\vspace{2mm}
				\hbox to 5.78in { {\em #3 \hfill #4} }
			}
		}
	\end{center}
	\vspace*{4mm}
}

\newcommand{\lecture}[4]{\handout{#1}{#2}{#3}{#4}{Lecture #1 (draft version 0)}}
\usepackage{amsthm}

\theoremstyle{definition}
\newtheorem{theorem}{Theorem}[section]
\newtheorem{corollary}[theorem]{Corollary}
\newtheorem{lemma}[theorem]{Lemma}
\newtheorem{observation}[theorem]{Observation}
\newtheorem{proposition}[theorem]{Proposition}
\newtheorem{definition}[theorem]{Definition}
\newtheorem{claim}[theorem]{Claim}
\newtheorem{remark}[theorem]{Remark}
\newtheorem{fact}[theorem]{Fact}
\newtheorem{assumption}[theorem]{Assumption}

% 1-inch margins, from fullpage.sty by H.Partl, Version 2, Dec. 15, 1988.
\topmargin 0pt
\advance \topmargin by -\headheight
\advance \topmargin by -\headsep
\textheight 8.9in
\oddsidemargin 0pt
\evensidemargin \oddsidemargin
\marginparwidth 0.5in
\textwidth 6.5in

\parindent 0in
\parskip 1.5ex
%\renewcommand{\baselinestretch}{1.25}

\usepackage[backend=bibtex,style=alphabetic,maxalphanames=4,minalphanames=4,sorting=nty]{biblatex}
\addbibresource{mybib.bib}


\usepackage{fancyhdr}
\pagestyle{fancy}
\lhead{Hodge Theory Lecture Notes}
\rhead{\thepage}
%\cfoot{center of the footer!}
\renewcommand{\headrulewidth}{0.4pt}
\renewcommand{\footrulewidth}{0.4pt}

\usepackage{hyperref}

\hypersetup{
	colorlinks=true,
	linkcolor=Blue,
	filecolor=YellowOrange,
	citecolor = WildStrawberry,      
	urlcolor=cyan,
}


\begin{document}
	
	\lecture{4 --- 08, 15, 2024}{Spring 2024}{}{Scribe: Yi Li}
	\section{Overview}
	The topics in today's lecture are: 
	\begin{enumerate}
		\item Proof of the Hodge structures in family satisfies the Griffiths transversality condition,
		\item We will construct the Higgs metric (not necessarily positive definition) and Hodge metric (positive definite) on the Hodge bundle,
		\item We will give a geometric discription of the Higgs field (which is actually the Kodaira-Spencer map),
		\item We will construct the compact dual, period domain, period mapping and we will introduce some basic properties about period mapping and period domain,
		\item We will study the curvature property on the period domain. As an application, we will show the moduli space of Calabi-Yau manifold is hyperbolic.
	\end{enumerate}

	\section{The Griffiths transversality theorem}
	We first introduce the Cartan-Lie formula, which will be used in the differential geometric proof of Griffiths transversality theorem.
	
	\begin{theorem}[Cartan-Lie formula, see \cite{Voisin1}, Proposition 9.14]
	Let $\pi:{X}\to \Delta$ be a smooth family (proper submersion), for any section $\sigma \in R^k f_* (\mathbb{C}) (U)$ there exist a smooth $\Omega \in \mathcal{A}^k(\mathcal{X})$ such that 
	\begin{enumerate}
		\item $\Omega|_{X_t}$ is d-closed,
		\item $\sigma(t)  = [\Omega|_{X_t}]$.
	\end{enumerate}
	Moreover $$\nabla_{u} \sigma(0) = [\iota_{v} d \Omega|_{X_0}],$$ where $v$ is a lift of $u$ (such that $\pi_* v = u$)  (e..g we can pick $u = \frac{\partial }{\partial  t}$).
	
	\end{theorem}
	\begin{proof}
	
	\end{proof}
	
	Now we can prove the Griffith transversality theorem
	\begin{theorem}[Griffith's Transversality theorem]
		Let $f:X\to S$ be smooth family of projective (algebraic) variety 
		(1) (The holomorphicity of the Hodge filtration bundle) The vector spaces $F^pH^k(X_t, \mathbb{C}) = \bigoplus_{r\ge p}{H}^{r,k-r}$ fit together into a holomorphic subbundle $F^p$.
		(2)(The transversality property) The filtration of holomorphic subbundle satisfies the Griffiths transversality condition
		$$\nabla : \mathcal{V}^{p,q}\to \mathcal{A}^{1,0}(\mathcal{V}^{p-1,q})\oplus \mathcal{A}^{1}(\mathcal{V}^{p,q})\oplus \mathcal{A}^{0,1}(\mathcal{V}^{p,q-1})$$	
	\end{theorem}
	\begin{proof}
	We will divide the proof into several steps:
	
	\textbf{Step 1: Hodge (p,q)-bundle is smooth subbundle of the Hodge bundle}
	
	The proof of this part need to use the theorem of Kodaira and Spencer. TODO
	
	\textbf{Step 2: Prove $\nabla _{\frac{\partial }{\partial  t}}: \mathcal{V}^{p,q}\to \mathcal{V}^{p,q}\oplus \mathcal{V}^{p-1,q+1}$ }
	
	We apply the Cartan-Lie formula so that given a smooth section $\sigma \in \Gamma(\mathcal{V}^{p,q})\subset  \Gamma(R^kf_* (\mathbb{C}))$ we can apply the Cartan-Lie formula, and find some smooth $\Omega \in \mathcal{A}^{k}(X)$ such that
	
	(1) $\Omega|_{X_t}$ is d-closed,
	
	(2) $[\Omega|_{X_t}] = \sigma(t) \in \mathcal{A}^{p,q}(X_t)$ , and 
	
	(3) $\nabla_{\frac{\partial }{\partial t}} \sigma (0 ) =  [\iota_v d \Omega|_{X_0}]$, here $v$ is a lifting of $\frac{\partial }{\partial t}$ says $f_* (v) =  \left(\frac{\partial }{\partial t}\right)$.
	
	Since the interior product (contraction satisfies the Libniz rule) so that $$\bar{\partial} (\iota_v \Omega) =  - \iota_v  \bar{\partial}\Omega +  \iota_{ \bar{\partial} v} \Omega$$if we write $$ \nabla _{\frac{\partial  }{\partial t}}  \sigma(0) = [\iota_v d \Omega |_{X_0}] = [\iota_v  (\partial+ \bar{\partial} )\Omega|_{X_0}] =  [\iota_v  \partial \Omega |_{X_0}] + [\bar{\partial} (\iota_v \Omega)|_{X_0} ]-  [\iota_{\bar{\partial} v} \Omega|_{X_0}]  =  [\iota_v \partial \Omega|_{X_0} ] - [\iota_{\bar{\partial} v}\Omega|_{X_0}]$$
	(Note that we apply the The ddbar lemma, thus $\bar{\partial}(\iota_v \Omega)$ is also $d$-exact)
	
	
	Comparing the type we note that:
	
	(a) $[\iota_v  \partial \Omega|_{X_0}] \in \mathcal{A}^{p,q}(X_0)$ (indeed we have $\Omega |_{X_0}\in \mathcal{A}^{p,q}(X_0)$ so that $\partial \Omega|_{X_0} \in \mathcal{A}^{p+1,q}(X_0)$ and therefore by contracting with the holomorphic tangent vector $v$ we get $[\iota_v \partial \Omega|_{X_0}] \in \mathcal{A}^{p,q}(X_0)$),
	
	(b) $[\iota_{\bar{\partial} v} \Omega|_{X_0}]\in \mathcal{A}^{p-1,q+1}(X_0)$ (indeed $\bar{\partial} v \in \mathcal{A}^{0,1}(\mathcal{T}_X)$ and contracting it we get $\iota_{\bar{\partial} v}\Omega|_{X_0} \in \mathcal{A}^{p-1,q+1}(X_0)$). 
	
	(Note that the der and derbar operator commute with the pull back iff the map is holomorphic we can commute the restriction with the differentiation.)
	
	So that $$\nabla _{\frac{\partial }{\partial  t}}: \mathcal{V}^{p,q}\to \mathcal{V}^{p,q}\oplus \mathcal{V}^{p-1,q+1}, \sigma  \mapsto  \nabla_{\frac{\partial }{\partial t}} \sigma$$
	
	
	\textbf{Step 3: Prove $\nabla_{\frac{\partial }{{\partial} \bar{t}}}:\mathcal{V}^{p,q}\to \overline{\mathcal{V}^{p,q}\oplus \mathcal{V}^{p+1,q-1}}$}
	
	We claim
	$$\nabla_{\frac{\partial }{{\partial} \bar{t}}}:\mathcal{V}^{p,q}\to {\mathcal{V}^{p,q}\oplus \mathcal{V}^{p+1,q-1}}$$
	
	The proof of this part need to use the parallel of the polarization. Since the polarization $Q$ is $\nabla$-parallel $$dQ(a,\bar{b}) =  Q(\nabla  a ,\bar{b}) + Q(a,\nabla \bar{b})$$
	eating the vector $\frac{\partial }{\partial t}$ we get $$ dQ(a,\bar{b}) (\frac{\partial }{\partial t}) = \frac{\partial }{\partial t} Q(a,\bar{b})  =  Q(\nabla_{\frac{\partial }{\partial t}} a  ,\bar{b}) + Q(a,\overline{ \nabla_{\frac{\partial }{\partial t}} b})  $$
	Since $Q(a,\bar{b}) \equiv 0$ for $a \in \mathcal{V}^{p,q}$ and $b \in \mathcal{V}^{r,s}$ unless $(p,q) = (r,s)$, consequently $$ Q\left(\nabla _{\frac{\partial }{\partial t }}a,\bar{b}\right) =  -Q(a, \overline{\nabla _{\frac{\partial }{\partial \bar{t}}}b})$$
	$$\nabla_{\frac{\partial }{\partial \overline{t}}}b \in \mathcal{V}^{p,q}\oplus { \mathcal{V}^{p+1,q-1}}$$
	
	Combine them together thus
	$$\nabla : \mathcal{V}^{p,q}\to \mathcal{A}^{1,0}(\mathcal{V}^{p-1,q})\oplus \mathcal{A}^{1}(\mathcal{V}^{p,q})\oplus \mathcal{A}^{0,1}(\mathcal{V}^{p,q-1})$$
	
	\end{proof}
	\section{Construction of the Hodge metric and Higgs metric}
	
	\section{Geometric interpretation of the Higgs field using Kodaira-Spencer map}
	
	\section{Construction of period domain (as homogenuous space)}
	
	\section{Holomorphicity of the period domain}
	
	\section{Tangent space of the period domain}
	
	\section{Tangent bundle of the period domain}
	
	\section{Horizental tangent bundle of the period domain}
	
	\section{Curvature properties}
	
	\section{Hyperbolicity on the moduli space of Calabi-Yau manifolds}
	
%	
%	\section{The fiber of the Moishezon morphism is again Moishezon}
%	
%	\begin{theorem}[The fiber of the Moishezon morphism is again Moishezon, see \cite{Moishezonmorphism}, Corollary 16]\label{Moishezon-morphism}
%	
%	\end{theorem}
%	
%	
%	\begin{remark}
%	\cite{Moishezonmorphism} 
%	\end{remark}
%	
%	\begin{proof}
%		
%	\end{proof}
%
%	\begin{center}
%		\begin{tikzcd}
%			&&& {H^0(\mathcal{X},R^2\pi_*\mathcal{O}_{\mathcal{X}})} \\
%			{} & {H^1(\mathcal{X},\mathcal{O}_{\mathcal{X}}^*)} & {H^2(\mathcal{X},\mathbb{Z})} & {H^2(\mathcal{X},\mathcal{O}_{\mathcal{X}})} & {} \\
%			{} & {H^1(X_s,\mathcal{O}_{X_s}^*)} & {H^2(X_s,\mathbb{Z})} & {H^2(X_s,\mathcal{O}_{X_s})} & {} \\
%			&&& {R^2\pi_*\mathcal{O}_{\mathcal{X}}(s)}
%			\arrow["\cong", from=1-4, to=2-4]
%			\arrow[from=2-1, to=2-2]
%			\arrow[from=2-2, to=2-3]
%			\arrow[from=2-2, to=3-2]
%			\arrow[from=2-3, to=2-4]
%			\arrow[from=2-3, to=3-3]
%			\arrow[from=2-4, to=2-5]
%			\arrow[from=2-4, to=3-4]
%			\arrow[from=3-1, to=3-2]
%			\arrow[from=3-2, to=3-3]
%			\arrow[from=3-3, to=3-4]
%			\arrow[from=3-4, to=3-5]
%			\arrow["\cong", from=3-4, to=4-4]
%		\end{tikzcd}
%	\end{center}
%
%	\begin{figure}[H]
%		\centering
%		\includegraphics[width=0.85\linewidth]{"Moishezon morphism definition"}
%		\caption{Comparison between different definitions for Moishezon morphism}
%		\label{fig:moishezon-morphism-definition}
%	\end{figure}
%	
	\printbibliography	
	
\end{document}
