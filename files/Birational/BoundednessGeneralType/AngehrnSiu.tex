\documentclass[11pt]{article}
\usepackage[dvipsnames]{xcolor}
\usepackage{latexsym}
\usepackage{amsmath}
\usepackage{mathrsfs}
\usepackage{tikz-cd}
\usepackage{amssymb}
\usepackage{amsthm}
\usepackage{epsfig}
\usepackage{graphicx}
\usepackage{float}
\newcommand{\handout}[5]{
	\noindent
	\begin{center}
		\framebox{
			\vbox{
				\hbox to 5.78in { {\bf Birational Geometry Reading Notes} \hfill #2 }
				\vspace{4mm}
				\hbox to 5.78in { {\Large \hfill #5  \hfill} }
				\vspace{2mm}
				\hbox to 5.78in { {\em #3 \hfill #4} }
			}
		}
	\end{center}
	\vspace*{4mm}
}

\newcommand{\lecture}[4]{\handout{#1}{#2}{#3}{#4}{Note #1 (draft version 0)}}
\usepackage{amsthm}

\theoremstyle{definition}
\newtheorem{theorem}{Theorem}[section]
\newtheorem{corollary}[theorem]{Corollary}
\newtheorem{lemma}[theorem]{Lemma}
\newtheorem{observation}[theorem]{Observation}
\newtheorem{proposition}[theorem]{Proposition}
\newtheorem{definition}[theorem]{Definition}
\newtheorem{claim}[theorem]{Claim}
\newtheorem{remark}[theorem]{Remark}
\newtheorem{fact}[theorem]{Fact}
\newtheorem{assumption}[theorem]{Assumption}

% 1-inch margins, from fullpage.sty by H.Partl, Version 2, Dec. 15, 1988.
\topmargin 0pt
\advance \topmargin by -\headheight
\advance \topmargin by -\headsep
\textheight 8.9in
\oddsidemargin 0pt
\evensidemargin \oddsidemargin
\marginparwidth 0.5in
\textwidth 6.5in

\parindent 0in
\parskip 1.5ex
%\renewcommand{\baselinestretch}{1.25}

\usepackage[backend=bibtex,style=alphabetic,maxalphanames=4,minalphanames=4,sorting=nty]{biblatex}
\addbibresource{mybib.bib}


\usepackage{fancyhdr}
\pagestyle{fancy}
\lhead{Birational Geometry Reading Notes}
\rhead{\thepage}
%\cfoot{center of the footer!}
\renewcommand{\headrulewidth}{0.4pt}
\renewcommand{\footrulewidth}{0.4pt}

\usepackage{hyperref}

\hypersetup{
	colorlinks=true,
	linkcolor=Blue,
	filecolor=YellowOrange,
	citecolor = WildStrawberry,      
	urlcolor=cyan,
}


\begin{document}
	
	\lecture{1 --- 08, 24, 2024}{Summer 2024}{}{Yi Li}
	\section{Introduction}
	The aim of this note is to prove the theorem of Angehrn and Siu. The theorem gave a quadratic result for global generation of the adjoint bundle. The major difficulty in the proof of the theorem is how to construct an effective $\mathbb{Q}$-divisor $D$ s.t. (a) the log canonical threshold $\text{LCT}(D,x) = 1$, and (b) $x$ is the isolated point the LC locus. Once we successfully construct such divisor $D$, the proof easily follows by appling the Nadal vanishing theorem to the exact sequence associated to the multiplier ideal sheaf $\mathcal{J}_D$. 
	
	Let us briefly sketch the idea on how to produce divisor $D$ above: We start our construction by observing that the multiplier ideal will vanish at a given smooth point if the multiplicity of the divisor $D_1$ is sufficient large. We then inductively construct a sequence of divisors $D_i$ such that the vanishing locus will strictly decrease $$Z_1= V(\mathcal{J}(D_1))\supsetneq Z_2 = V(\mathcal{J}(D_2)) \supsetneq \ldots,$$since the dimension strictly decrease, there is some divisor $D_k$ such that $V(\mathcal{J}(D_k))$ is a point. The major problem that will appear in the induction process is that although $x\in X$ is smooth point of $X$ but it can be singular point of $Z_1$. The idea of Angehrn and Siu is taking a curve passing through $x$ such that all the nearby points on the curve are smooth points and therefore it's possible to construct some divisor $D_{2,t}$, then they use a limiting arguement and construct $D_{2,t}\to D_2$ and check that $D_2$ is the one that we hope. 
	
	This note is organized as follows: In Section 2. we will state the main theorem with some applications. In Section 3. we will summarize the technical tools needed in the proof. In Section 4. we will prove the main results. We will end this note by showing some further development after Angehrn and Siu. 
	
	The major references of this note are \cite{DemaillyAG}, \cite{Lazars2Vec}. 
	
	\section{Main results}
	
	\begin{theorem}
		Let $X$ be a smooth projective variety of dimension $n$, and let $L$ be an ample divisor on $X$. Fix a point $x \in X$, and assume that
		$$
		\left(L^{\operatorname{dim} Z} \cdot Z\right)>\binom{n+1}{2}^{\operatorname{dim} Z} = M^{\dim Z}
		$$
		for every irreducible subvariety $Z \subseteq X$ passing through $x$ (including of course $X$ itself $)$. Then $K_X+L$ is free at $x$, i.e. $\mathcal{O}_X\left(K_X+L\right)$ has a section which does not vanish at $x$.
		
		In particular, if $L \equiv_{\text {num }} m A$ for some $m>\binom{n+1}{2}$ and some ample $A$, then $K_X+L$ is free. 

	\end{theorem}

	
	\section{Technical tools}
	\subsection{Producing multiplier ideal vanishing at given point}
	The theorem below shows that the higher the multiplicity of a divisor, the deeper the multiplier ideal sheaf associated to the divisor.
	\begin{lemma}
		Let $X$ be smooth variety with $D\ge 0$ and effective $\mathbb{R}$-divisor, then 
		
		\begin{enumerate}
			\item If $\text{mult}_x D<1$ then $\mathcal{J}(D)_x = \mathcal{O}_{X,x}$,
			\item If $\text{mult}_x D \geq n$, then $\mathcal{J}(D) \subset \mathfrak{m}_x$, where $\mathfrak{m}_x$ is the corresponding maximal ideal.
		\end{enumerate}
	\end{lemma}
	\begin{proof}
		
	\end{proof}
	\subsection{Using divisor with large volume to produce a section with high multiplicity}
	\begin{lemma}[Divisor with high multiplicity in the linear system with large volume]\label{highmulti}
		Let $x \in V$ be a smooth point on an irreducible projective variety of dimension $d$, $0<a \in \mathbb{Q}$ and $A$ an ample Cartier divisor on $V$ such that the volume is lowerbound $A^d = \text{Vol}(A)>a^d$. 
		
		Then, for any $k \gg 0$, there exists a divisor $A_k \in|k A|$ such that $\operatorname{mult}_x(A)>k a$.
	\end{lemma}
	\begin{proof}
		
	\end{proof}
	\subsection{Cutting down the LC locus using some divisor with high multiplicity}
	\begin{lemma}[Cutting down the LC locus using divisor with high multiplicity]\label{cutLC}
		Let $X$ be a smooth variety and $x \in X$ a fixed point. Consider an effective $\mathbf{Q}$-divisor $D$ on $X$ with $c(D ; x)=1$, and suppose that $Z=\operatorname{LC}(D ; x)$ is irreducible of dimension $d$ at $x$. 
		
		Fix a general smooth point $y \in Z$, and let $B$ be any effective $\mathbb{Q}$-divisor on $X$, with $Z \nsubseteq \operatorname{Supp} B$. Assume that
		$$
		\operatorname{mult}_y\left(Z ; B_Z\right)>d=\operatorname{dim} Z,
		$$
		where $B_Z$ denotes the restriction of $B$ to $Z$, and as indicated the multiplicity is computed on $Z$. 
		
		Then 
		\begin{enumerate}
			\item For $0<\varepsilon \ll 1$,
			$$
			\mathcal{J}(X,(1-\varepsilon) D+B)
			$$
			is non-trivial at $y$.
			\item If moreover $\mathcal{J}(X, D+B)=\mathcal{J}(X, D)$ away from $Z$, then
			$$
			Z^{\prime}= \operatorname{Zeroes}(\mathcal{J}((1-\varepsilon) D+B))
			$$
			is a proper subvariety of $Z$ in a neighborhood of $y$.
		\end{enumerate} 
	\end{lemma}
	\begin{proof}
		
	\end{proof}
	\subsection{Peturbation of LC locus}
	\begin{lemma}	
		Let $X$ be a nonsingular variety, $x \in X$ a fixed point and $D$ an effective $\mathrm{Q}$-divisor whose support contains $x$. Then there exists an effective $\mathbf{Q}$-divisor $E$ such that
		$$
		\mathrm{LC}((D+t E) ; x) \subseteq X
		$$
		is irreducible at $x$ for all $0<t \ll 1$. 
		
		Moreover if $X$ is projective and $L$ is any ample divisor, then one can take $E \equiv_{\text {num }} p L$ for some $p>0$.
	\end{lemma}
	\begin{proof}
		
	\end{proof}
	\section{Proof of the theorem}
		\subsection{Construct divisor $D_1$}
		\begin{proof}
			
		\end{proof}
		\subsection{Construct divisor $D_i$ inductively}
		
		Let $Z = V(\mathcal{J}(D_1))$ and $d= \dim Z$. We need to inductively find a sequence of divisors $\{D_i\}$, such that $$Z= Z_1 =V(\mathcal{J}(D_1))\supsetneq V(\mathcal{J}(D_2))\supsetneq \ldots,$$We hope to repeat the construction process like $D_1$. The provblem is $x\in Z_1$ needs not to be smooth point of $Z_1$.  
		
		The idea to solve this problem is to use Lemma \ref{cutLC}. And we need to find some divisor $B_0$ s.t. 
		(a) $Z \nsubseteq \operatorname{Supp} B$,
		(b) $\operatorname{mult}_y\left(Z ; B_Z\right)>d=\operatorname{dim} Z,$ where $B_Z$ denotes the restriction of $B$ to $Z$, and as indicated the multiplicity is computed on $Z$,
		(c) $\mathcal{J}(X, D+B)=\mathcal{J}(X, D)$ away from $Z$.
		
		
		\begin{proof}
			We divide the proof into 3 steps:
			
			Step 1. (Construction of the limiting divisor $\bar{A}_0$ on $Z$)
			
			Step 2. (Lifting the divisors $\bar{A}_t$ onto $X$)
			
			Step 3. (Apply the cutting down of LC locus Lemma \ref{cutLC} to $B_0$)
			
			
			
		\end{proof}
		\subsection{Finish the prood of the main theorem}
		\begin{proof}
			Now we find a $\mathbb{Q}$-effective divisor $D$ such that $$D \equiv \lambda L , \quad \lambda<1$$and the log canonical threshold $\text{LCT}(D,x) = 1$ with $x\in X$ being the isolated point of the log canonical locus. 
			
			Consider the following short exact sequence induced by the multiplier ideal sheaf $\mathcal{J}(D)$ $$0 \to \mathcal{J}(D) \to \mathcal{O}_X(K_X+L) \to \mathcal{O}_Z(K_X+L)\to 0,$$where $Z = V(\mathcal{J})$. It will induce $$0\to \ldots\to H^0(X,\mathcal{O}_X(K_X+L)) \to H^0(Z,\mathcal{O}_Z(K_X+L))\to H^1(X,\mathcal{J}(D)\otimes \mathcal{O}_X(K_X+L)) \to  \ldots.$$
			Since $L$ is sufficient positive and $\lambda<1$, by Nadel vanishing theorem  $H^1(X,\mathcal{J}(D) \otimes \mathcal{O}_X(K_X+L)) = 0$. Thus the section on $Z$ can be lifted to $X$ i.e. $$H^0(X,\mathcal{O}_X(K_X+L)) \twoheadrightarrow H^0(Z,\mathcal{O}_Z(K_X+L)).$$
			Recall that the log canonical locus $$\text{LC} (D,x) = V(\mathcal{J}(\text{LCT}(D,x)\cdot D))$$since $\text{LCT}(D,x) = 1$ this means $Z$ is precisely the log canonical locus of $D$. By assumption $x$ is the isolated point of the LC locus, the section $s \in H^0(Z,\mathcal{O}_Z(K_X+D))$ can take any value at $x\in X$, in particular $s$ can choose to be a section that does not vanish at $x$. Lift it to $\widetilde{s} \in H^0(X,\mathcal{O}_X(K_X+D))$ gives the section we want.
			
		\end{proof}
		
	\section{Further remarks}
	The aim of this section is to introduce some further study of theorem Angehrn and Siu. 
	
	\subsection{Remarks about the Fujita conjecture}
	\subsection{Application in the birational boundedness problem}
	\subsection{Relation with effective base point free theorem}
%	
%	\section{The fiber of the Moishezon morphism is again Moishezon}
%	
%	\begin{theorem}[The fiber of the Moishezon morphism is again Moishezon, see \cite{Moishezonmorphism}, Corollary 16]\label{Moishezon-morphism}
%		
%	\end{theorem}
%	
%	
%	\begin{remark}
%		\cite{Moishezonmorphism} 
%	\end{remark}
%	
%	\begin{proof}
%		
%	\end{proof}
%	
%	\begin{center}
%		\begin{tikzcd}
%			&&& {H^0(\mathcal{X},R^2\pi_*\mathcal{O}_{\mathcal{X}})} \\
%			{} & {H^1(\mathcal{X},\mathcal{O}_{\mathcal{X}}^*)} & {H^2(\mathcal{X},\mathbb{Z})} & {H^2(\mathcal{X},\mathcal{O}_{\mathcal{X}})} & {} \\
%			{} & {H^1(X_s,\mathcal{O}_{X_s}^*)} & {H^2(X_s,\mathbb{Z})} & {H^2(X_s,\mathcal{O}_{X_s})} & {} \\
%			&&& {R^2\pi_*\mathcal{O}_{\mathcal{X}}(s)}
%			\arrow["\cong", from=1-4, to=2-4]
%			\arrow[from=2-1, to=2-2]
%			\arrow[from=2-2, to=2-3]
%			\arrow[from=2-2, to=3-2]
%			\arrow[from=2-3, to=2-4]
%			\arrow[from=2-3, to=3-3]
%			\arrow[from=2-4, to=2-5]
%			\arrow[from=2-4, to=3-4]
%			\arrow[from=3-1, to=3-2]
%			\arrow[from=3-2, to=3-3]
%			\arrow[from=3-3, to=3-4]
%			\arrow[from=3-4, to=3-5]
%			\arrow["\cong", from=3-4, to=4-4]
%		\end{tikzcd}
%	\end{center}
%	
%	\begin{figure}[H]
%		\centering
%		\includegraphics[width=0.85\linewidth]{"Moishezon morphism definition"}
%		\caption{Comparison between different definitions for Moishezon morphism}
%		\label{fig:moishezon-morphism-definition}
%	\end{figure}
%	
	\printbibliography	
	
\end{document}
