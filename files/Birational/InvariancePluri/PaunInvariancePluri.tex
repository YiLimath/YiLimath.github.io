\documentclass[11pt]{article}
\usepackage[dvipsnames]{xcolor}
\usepackage{latexsym}
\usepackage{amsmath}
\usepackage{mathrsfs}
\usepackage{tikz-cd}
\usepackage{amssymb}
\usepackage{amsthm}
\usepackage{epsfig}
\usepackage{graphicx}
\usepackage{float}
\newcommand{\handout}[5]{
	\noindent
	\begin{center}
		\framebox{
			\vbox{
				\hbox to 5.78in { {\bf Invariance of Plurigenera Reading Notes} \hfill #2 }
				\vspace{4mm}
				\hbox to 5.78in { {\Large \hfill #5  \hfill} }
				\vspace{2mm}
				\hbox to 5.78in { {\em #3 \hfill #4} }
			}
		}
	\end{center}
	\vspace*{4mm}
}

\newcommand{\lecture}[4]{\handout{#1}{#2}{#3}{#4}{Lecture #1 (draft version)}}
\usepackage{amsthm}

\theoremstyle{definition}
\newtheorem{theorem}{Theorem}[section]
\newtheorem{corollary}[theorem]{Corollary}
\newtheorem{lemma}[theorem]{Lemma}
\newtheorem{observation}[theorem]{Observation}
\newtheorem{proposition}[theorem]{Proposition}
\newtheorem{definition}[theorem]{Definition}
\newtheorem{claim}[theorem]{Claim}
\newtheorem{remark}[theorem]{Remark}
\newtheorem{fact}[theorem]{Fact}
\newtheorem{assumption}[theorem]{Assumption}

% 1-inch margins, from fullpage.sty by H.Partl, Version 2, Dec. 15, 1988.
\topmargin 0pt
\advance \topmargin by -\headheight
\advance \topmargin by -\headsep
\textheight 8.9in
\oddsidemargin 0pt
\evensidemargin \oddsidemargin
\marginparwidth 0.5in
\textwidth 6.5in

\parindent 0in
\parskip 1.5ex
%\renewcommand{\baselinestretch}{1.25}

\usepackage[backend=bibtex,style=alphabetic,maxalphanames=4,minalphanames=4,sorting=nty]{biblatex}
\addbibresource{mybib.bib}


\usepackage{fancyhdr}
\pagestyle{fancy}
\lhead{Birational Geometry Reading Notes}
\rhead{\thepage}
%\cfoot{center of the footer!}
\renewcommand{\headrulewidth}{0.4pt}
\renewcommand{\footrulewidth}{0.4pt}

\usepackage{hyperref}

\hypersetup{
	colorlinks=true,
	linkcolor=Blue,
	filecolor=YellowOrange,
	citecolor = WildStrawberry,      
	urlcolor=cyan,
}


\begin{document}
	
	\lecture{1 --- 25, 08, 2024}{Summer 2024}{}{Yi Li}
	\section{Overview}
	
	This note will give a comprehensive introduction to Paun's proof on invariance of plurigenera \cite{PaunSiu} and also \cite{DemaillyAG}. The main result in the paper shows a extension theorem for the pluricanonical section twisted with some positive line bundles. As a direct consequence the invariance of plurigenera holds for smooth projective family with general type assumption.

	\subsection{Idea of the proof}
	As we will see later, the invariance of plurigenera problem can be easily reduce to problem of extension of pluricanonical section $u \in H^0(X_0,mK_{X_0})$. To extend it to $\widetilde{u} \in H^0(X,mK_X)$, by Ohsawa-Takegoshi, it's suffices to find a metric $h_{(m-1)}$ on $(m-1)K_X$ such that the curvature current is semi-positive and most importantly $\int_{X_0} |u|^2_{\omega  \otimes h_{(m-1)}}<+\infty$. The rest part of this subsection will sketch the idea how to construct the metric on the pluricanonical bundle.
	
	The idea of Paun is to construct the metric on adjoint bundle $K_X+L$  using section on sequence of line bundles $F_k$, and take limit. (As a side remark we can set $L= (m-1)K_X$ in applications and it will come back the the extension of pluricanonical section problem).
	
	We divide the construction process into 2 steps:
	
	In step 1, we try to construct the metric on $F_k= k(K_X+L)+A$ by inserting some sufficient ample divisor $A$, assume $A$ is sufficient ample so that it is global generated. What nice about this setting is the tautological metric on $A$ defined using those generators $(u_j)$ will have integrable condition needed in the Ohsawa-Takegoshi extension theorem (recall that the only singularity of the tautological metric appears on the domoninator $\frac{1}{\sum_j |u_j|^2}$, thanks to base point freeness). We then inductively construct metric on $F_k$. Assume by induction we already find a set of global section $(\widetilde{u}_j^{(k)})$  of $F_{k-1}$ with the restriction $u_j^{(k)} = \widetilde{u}^{(k)}_j |_{X_0} = \sigma^k u_j$, we can then define the tautological metric on $F_k = F_{k-1}+K_X+L$ under trivialization as $\frac{|\xi|^2}{\sum_j |\widetilde{u}_j^{(k-1)}|^2} e^{-\varphi_L} dV_\omega$ if we substitute $\xi= u_j$ it will clearly be integrable on $X_0$. Thus by Ohsawa-Takegoshi extension theorem we can find $(\widetilde{u}_j)$ on $F_k$, with restriction $\widetilde{u}_j^{(k)}|_{X_0} = u_j^k$. 
	
	In step 2, we try to take roots and limit on the singular Hermitian metric. First, by the induction process given above, we can define a sequence of singular Hermitian metric $\varphi_k$ on $\frac{1}{k}F_k= K_X+L+ \frac{1}{k} A$ (by taking roots). Second, taking $k\to \infty$, there is a hope to find a limiting metric $\varphi$ on the $K_X+L$ satisfies all the nice conditions we want. 
	\subsection{Some important ideas in the proof}
	We summarize some essential idea in the proof here: 
	\section{Preliminaries}
	\subsection{Construct metric using global sections}
	
	\subsection{Ohsawa-Takegoshi extension theorem}
	
	\subsection{Upper semi-continuity theorem}
	
	\subsection{Reduce the invariance of plurigenera problem to extension of pluricanonical form problem}
	
	\section{Paun's extension theorem of pluricanonical form}
	

%	
%	\section{The fiber of the Moishezon morphism is again Moishezon}
%	
%	\begin{theorem}[The fiber of the Moishezon morphism is again Moishezon, see \cite{Moishezonmorphism}, Corollary 16]\label{Moishezon-morphism}
%	
%	\end{theorem}
%	
%	
%	\begin{remark}
%	\cite{Moishezonmorphism} 
%	\end{remark}
%	
%	\begin{proof}
%		
%	\end{proof}
%
%	\begin{center}
%		\begin{tikzcd}
%			&&& {H^0(\mathcal{X},R^2\pi_*\mathcal{O}_{\mathcal{X}})} \\
%			{} & {H^1(\mathcal{X},\mathcal{O}_{\mathcal{X}}^*)} & {H^2(\mathcal{X},\mathbb{Z})} & {H^2(\mathcal{X},\mathcal{O}_{\mathcal{X}})} & {} \\
%			{} & {H^1(X_s,\mathcal{O}_{X_s}^*)} & {H^2(X_s,\mathbb{Z})} & {H^2(X_s,\mathcal{O}_{X_s})} & {} \\
%			&&& {R^2\pi_*\mathcal{O}_{\mathcal{X}}(s)}
%			\arrow["\cong", from=1-4, to=2-4]
%			\arrow[from=2-1, to=2-2]
%			\arrow[from=2-2, to=2-3]
%			\arrow[from=2-2, to=3-2]
%			\arrow[from=2-3, to=2-4]
%			\arrow[from=2-3, to=3-3]
%			\arrow[from=2-4, to=2-5]
%			\arrow[from=2-4, to=3-4]
%			\arrow[from=3-1, to=3-2]
%			\arrow[from=3-2, to=3-3]
%			\arrow[from=3-3, to=3-4]
%			\arrow[from=3-4, to=3-5]
%			\arrow["\cong", from=3-4, to=4-4]
%		\end{tikzcd}
%	\end{center}
%
%	\begin{figure}[H]
%		\centering
%		\includegraphics[width=0.85\linewidth]{"Moishezon morphism definition"}
%		\caption{Comparison between different definitions for Moishezon morphism}
%		\label{fig:moishezon-morphism-definition}
%	\end{figure}
%	
	\printbibliography	
	
\end{document}
