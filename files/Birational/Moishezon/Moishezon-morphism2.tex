\documentclass[11pt]{article}
\usepackage{latexsym}
\usepackage{amsmath}
\usepackage{mathrsfs}
\usepackage{amssymb}
\usepackage{amsthm}
\usepackage{epsfig}
\usepackage{graphicx}
\usepackage{float}
\newcommand{\handout}[5]{
	\noindent
	\begin{center}
		\framebox{
			\vbox{
				\hbox to 5.78in { {\bf Moishezon morphism reading seminars} \hfill #2 }
				\vspace{4mm}
				\hbox to 5.78in { {\Large \hfill #5  \hfill} }
				\vspace{2mm}
				\hbox to 5.78in { {\em #3 \hfill #4} }
			}
		}
	\end{center}
	\vspace*{4mm}
}

\newcommand{\lecture}[4]{\handout{#1}{#2}{#3}{Scribe: #4}{Lecture #5}}

\newtheorem{theorem}{Theorem}
\newtheorem{corollary}[theorem]{Corollary}
\newtheorem{lemma}[theorem]{Lemma}
\newtheorem{observation}[theorem]{Observation}
\newtheorem{proposition}[theorem]{Proposition}
\newtheorem{definition}[theorem]{Definition}
\newtheorem{claim}[theorem]{Claim}
\newtheorem{fact}[theorem]{Fact}
\newtheorem{assumption}[theorem]{Assumption}

% 1-inch margins, from fullpage.sty by H.Partl, Version 2, Dec. 15, 1988.
\topmargin 0pt
\advance \topmargin by -\headheight
\advance \topmargin by -\headsep
\textheight 8.9in
\oddsidemargin 0pt
\evensidemargin \oddsidemargin
\marginparwidth 0.5in
\textwidth 6.5in

\parindent 0in
\parskip 1.5ex
%\renewcommand{\baselinestretch}{1.25}

\begin{document}
	
	\lecture{7 --- 05,30, 2024}{Fall 2023}{}{Yi Li}
	\section{Overview}
	Today we will continue our discussion on the paper on Moishezon morphisms. Last time, we proved that a Moishezon manifold admits a (strong) Hodge decomposition. We also showed that a Kähler Moishezon space with 1-rational singularity is automatically projective, and finally, we gave three different definitions for Moishezon morphism and proved that they are equivalent.
	
	(1) We will prove that a proper surjective morphism equipped with a relatively big line bundle is locally bimeromorphic to a projective morphism.
	
	(2) We will prove that if the base space is Moishezon, then the total space is Moishezon if and only if the morphism is Moishezon.
	
	(3) We will show that the restriction of the generic surjective morphism on the exceptional set is a Moishezon morphism.
	
	
	\section{Chow's lemma, Reducing Proper Morphism to Projective Morphisms}
	We will continue our discussion on the Moishezon morphism last time.
	
	Recall that we define a morphism to be Moishezon if it's bimeromorphic to a projective morphism:
	\begin{definition}[]
		
	\end{definition}
	
	\begin{theorem}[Chow's lemma, \cite{DasHacon}]
		
	\end{theorem}
	\begin{theorem}[Reducing Proper Morphism to Projective Morphisms, \cite{DasHacon}]
		Let $f: X \rightarrow S$ be a proper surjective morphism of analytic varieties, and let $L$ be a $f$-big line bundle on $X$ and $D$ a $\mathbb{Q}$-divisor. 
		
		Then over any relatively compact open subset $V \subset S$, there exists a proper bimeromorphic morphism $\alpha: W \rightarrow$ $f^{-1} V$ from a smooth analytic variety $W$ such that $\beta=\left.f\right|_{f^{-1} V} \circ \alpha: W \rightarrow V$ is a projective morphism and $\left(W, \alpha_*^{-1}\left(\left.D\right|_{f^{-1} V}\right)+\operatorname{Ex}(\alpha)\right)$ is a log smooth pair.
	\end{theorem}
	\begin{proof}
		
	\end{proof}
	\section{When the base is Moishezon then the total space is Moishezon iff the morphism is Moishezon}
	
	\begin{theorem}
		
	\end{theorem}
	\section{Several examples}
	
	
	\section{}
	%	Here is a subsubsection. You can use these as well.
	%	
	%	\subsection{Using Boldface}
	%	Make sure to use lots of boldface.
	%	
	%	\paragraph{Question:}
	%	How would you use boldface?
	%	
	%	\paragraph{Example:}
	%	This is an example showing how to use boldface to 
	%	help organize your lectures.
	%	
	%	
	%	\paragraph{Some Formatting.}
	%	Here is some formatting that you can use in your notes:
	%	\begin{itemize}
		%		\item {\em Item One} -- This is the first item.
		%		\item {\em Item Two} -- This is the second item.
		%		\item \dots and here are other items.
		%	\end{itemize}
	%	
	%	If you need to number things, you can use this style:
	%	\begin{enumerate}
		%		\item {\em Item One} -- Again, this is the first item.
		%		\item {\em Item Two} -- Again, this is the second item.
		%		\item \dots and here are other items.
		%	\end{enumerate}
	%	
	%	\paragraph{Bibliography.}
	
	%\bibliography{mybib}
	\bibliographystyle{alpha}
	
	
	
\end{document}
