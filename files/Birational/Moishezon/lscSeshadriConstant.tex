\documentclass[11pt]{article}
\usepackage[dvipsnames]{xcolor}
\usepackage{latexsym}
\usepackage{amsmath}
\usepackage{mathrsfs}
\usepackage{tikz-cd}
\usepackage{amssymb}
\usepackage{amsthm}
\usepackage{epsfig}
\usepackage{graphicx}
\usepackage{float}
\newcommand{\handout}[5]{
	\noindent
	\begin{center}
		\framebox{
			\vbox{
				\hbox to 5.78in { {\bf Projectivity Criterira Reading Seminars} \hfill #2 }
				\vspace{4mm}
				\hbox to 5.78in { {\Large \hfill #5  \hfill} }
				\vspace{2mm}
				\hbox to 5.78in { {\em #3 \hfill #4} }
			}
		}
	\end{center}
	\vspace*{4mm}
}

\newcommand{\lecture}[4]{\handout{#1}{#2}{#3}{#4}{Note #1 (draft version 0)}}
\usepackage{amsthm}

\theoremstyle{definition}
\newtheorem{theorem}{Theorem}[section]
\newtheorem{corollary}[theorem]{Corollary}
\newtheorem{lemma}[theorem]{Lemma}
\newtheorem{observation}[theorem]{Observation}
\newtheorem{proposition}[theorem]{Proposition}
\newtheorem{definition}[theorem]{Definition}
\newtheorem{claim}[theorem]{Claim}
\newtheorem{remark}[theorem]{Remark}
\newtheorem{fact}[theorem]{Fact}
\newtheorem{assumption}[theorem]{Assumption}

% 1-inch margins, from fullpage.sty by H.Partl, Version 2, Dec. 15, 1988.
\topmargin 0pt
\advance \topmargin by -\headheight
\advance \topmargin by -\headsep
\textheight 8.9in
\oddsidemargin 0pt
\evensidemargin \oddsidemargin
\marginparwidth 0.5in
\textwidth 6.5in

\parindent 0in
\parskip 1.5ex
%\renewcommand{\baselinestretch}{1.25}

\usepackage[backend=bibtex,style=alphabetic,maxalphanames=4,minalphanames=4,sorting=nty]{biblatex}
\addbibresource{mybib.bib}


\usepackage{fancyhdr}
\pagestyle{fancy}
\lhead{Birational Geometry Reading Seminars}
\rhead{\thepage}
%\cfoot{center of the footer!}
\renewcommand{\headrulewidth}{0.4pt}
\renewcommand{\footrulewidth}{0.4pt}

\usepackage{hyperref}

\hypersetup{
	colorlinks=true,
	linkcolor=Blue,
	filecolor=YellowOrange,
	citecolor = WildStrawberry,      
	urlcolor=cyan,
}


\begin{document}
	
	\lecture{2 --- 11, 09, 2024}{Fall 2024}{}{Yi Li}
	\section{Overview}
	Last time we proved the Seshadri projectivity criterion of Moishezon variety. 
	
	This time we first introduce the Chow-Barlet cycle space and prove a technical result which shows that we can approximate the Chow-Barlet cycle space by countable many projective morphism. Using this result, we can prove the main theorem of today's seminar that is some lower semi-continuity result on Seshadri constant.
	
	Let us brief sketch the idea of the proof. 
	\section{Questions from the previous time}
	
	\section{Chow-Barlet cycle space}
	
	In this Section, we will study the main technical tools 
	
	We can now state the result which allows to approximate the Chow-Barlet cycle space using countable many projective morphism
	
	\begin{theorem}[Approximation Chow-Barlet cycle space using countable many projective morphisms]
	Let $g: X \rightarrow S$ be a proper morphism of complex analytic spaces that is bimeromorphic to a projective morphism. Fix $m \in \mathbb{N}$. Then there are countably many diagrams of complex analytic spaces over $S$, 
	\begin{center}
		\begin{tikzcd} {C_i} & {W_i\times_SX} \\ {W_i} \arrow["{w_i}"', shift right, from=1-1, to=2-1] \arrow["{\sigma_i}"', shift right, from=2-1, to=1-1] \arrow[hook, from=1-1, to=1-2] \end{tikzcd}
	\end{center}
	indexed by $i \in I$, such that
	
	(1) the $w_i: C_i \rightarrow W_i$ are proper, of pure relative dimension 1 and flat over a dense, Zariski open subset $W_i^{\circ} \subset W_i$,
	
	(2) the fiber of $w_i$ over any $p \in W_i^{\circ}$ has multiplicity $m$ at $\sigma_i(p)$,
	
	(3) the $W_i$ are irreducible, the structure maps $\pi_i: W_i \rightarrow S$ are projective, and
	
	(4) the fibers over all the $W_i^{\circ}$ give all irreducible curves that have multiplicity $m$ at the marked point.
	
	\end{theorem}
	
	\begin{proof}
		We will divide the proof into several steps.
		
	\end{proof}
	
	
	\section{Lower semi-continuity of Seshadri constant (openness of projectivity)}
	
	The main result that will be proved in today's seminar is 	
	\begin{theorem}[Openness of projectivity over Euclidean open subset, see \cite{Seshardri}, Proposition 14.]
		
	\end{theorem}
	
	
	\begin{proof}
		
	\end{proof}
	
	
%	
%	\section{The fiber of the Moishezon morphism is again Moishezon}
%	
%	\begin{theorem}[The fiber of the Moishezon morphism is again Moishezon, see \cite{Moishezonmorphism}, Corollary 16]\label{Moishezon-morphism}
%	
%	\end{theorem}
%	
%	
%	\begin{remark}
%	\cite{Moishezonmorphism} 
%	\end{remark}
%	
%	\begin{proof}
%		
%	\end{proof}
%
%	\begin{center}
%		\begin{tikzcd}
%			&&& {H^0(\mathcal{X},R^2\pi_*\mathcal{O}_{\mathcal{X}})} \\
%			{} & {H^1(\mathcal{X},\mathcal{O}_{\mathcal{X}}^*)} & {H^2(\mathcal{X},\mathbb{Z})} & {H^2(\mathcal{X},\mathcal{O}_{\mathcal{X}})} & {} \\
%			{} & {H^1(X_s,\mathcal{O}_{X_s}^*)} & {H^2(X_s,\mathbb{Z})} & {H^2(X_s,\mathcal{O}_{X_s})} & {} \\
%			&&& {R^2\pi_*\mathcal{O}_{\mathcal{X}}(s)}
%			\arrow["\cong", from=1-4, to=2-4]
%			\arrow[from=2-1, to=2-2]
%			\arrow[from=2-2, to=2-3]
%			\arrow[from=2-2, to=3-2]
%			\arrow[from=2-3, to=2-4]
%			\arrow[from=2-3, to=3-3]
%			\arrow[from=2-4, to=2-5]
%			\arrow[from=2-4, to=3-4]
%			\arrow[from=3-1, to=3-2]
%			\arrow[from=3-2, to=3-3]
%			\arrow[from=3-3, to=3-4]
%			\arrow[from=3-4, to=3-5]
%			\arrow["\cong", from=3-4, to=4-4]
%		\end{tikzcd}
%	\end{center}
%
%	\begin{figure}[H]
%		\centering
%		\includegraphics[width=0.85\linewidth]{"Moishezon morphism definition"}
%		\caption{Comparison between different definitions for Moishezon morphism}
%		\label{fig:moishezon-morphism-definition}
%	\end{figure}
%	
	\printbibliography	
	
\end{document}
