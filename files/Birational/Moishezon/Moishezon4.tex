\documentclass[11pt]{article}
\usepackage{latexsym}
\usepackage{amsmath}
\usepackage{mathrsfs}
\usepackage{tikz-cd}
\usepackage{amssymb}
\usepackage{amsthm}
\usepackage{epsfig}
\usepackage{graphicx}
\usepackage{float}
\newcommand{\handout}[5]{
	\noindent
	\begin{center}
		\framebox{
			\vbox{
				\hbox to 5.78in { {\bf Moishezon morphism reading seminars} \hfill #2 }
				\vspace{4mm}
				\hbox to 5.78in { {\Large \hfill #5  \hfill} }
				\vspace{2mm}
				\hbox to 5.78in { {\em #3 \hfill #4} }
			}
		}
	\end{center}
	\vspace*{4mm}
}

\newcommand{\lecture}[4]{\handout{#1}{#2}{#3}{#4}{Lecture #1 (draft version)}}
\usepackage{amsthm}

\theoremstyle{definition}
\newtheorem{theorem}{Theorem}[section]
\newtheorem{corollary}[theorem]{Corollary}
\newtheorem{lemma}[theorem]{Lemma}
\newtheorem{observation}[theorem]{Observation}
\newtheorem{proposition}[theorem]{Proposition}
\newtheorem{definition}[theorem]{Definition}
\newtheorem{claim}[theorem]{Claim}
\newtheorem{remark}[theorem]{Remark}
\newtheorem{fact}[theorem]{Fact}
\newtheorem{assumption}[theorem]{Assumption}

% 1-inch margins, from fullpage.sty by H.Partl, Version 2, Dec. 15, 1988.
\topmargin 0pt
\advance \topmargin by -\headheight
\advance \topmargin by -\headsep
\textheight 8.9in
\oddsidemargin 0pt
\evensidemargin \oddsidemargin
\marginparwidth 0.5in
\textwidth 6.5in

\parindent 0in
\parskip 1.5ex
%\renewcommand{\baselinestretch}{1.25}

\usepackage[backend=bibtex,style=alphabetic,maxalphanames=4,minalphanames=4,sorting=nty]{biblatex}
\addbibresource{mybib.bib}


\usepackage{fancyhdr}
\pagestyle{fancy}
\lhead{Moishezon morphism reading seminars}
\rhead{\thepage}
%\cfoot{center of the footer!}
\renewcommand{\headrulewidth}{0.4pt}
\renewcommand{\footrulewidth}{0.4pt}


\begin{document}
	
	\lecture{4 --- 06, 06, 2024}{Spring 2024}{}{Yi Li}
	\section{Overview}
	Today we will continue our discussion on the paper Moishezon morphism. Recall that last time we proved that a proper surjective morphism equipped with a relatively big line bundle is locally bimeromorphic to a projective morphism. We also proved that if the base space is Moishezon, then the total space is Moishezon if and only if the morphism is Moishezon. And finally the restriction of the generic surjective morphism on the exceptional set is a Moishezon morphism.
	
	This time we will show:
	(1) The fiber of the Moishezon morphism is Moishezon,
	(2) The alternation property of the very big locus, general type locus, and Moishezon locus, which is either nowhere dense or contains some dense open subset.
	
	\section{The fiber of the Moishezon morphism is again Moishezon}
	We will start today's discussion by using \cite{Moishezonmorphism}, Lemma 15. to prove the following theorem
	\begin{theorem}[The fiber of the Moishezon morphism is again Moishezon, see \cite{Moishezonmorphism}, Corollary 16]\label{Moishezon-morphism}
	The fibers of a proper, Moishezon morphism are Moishezon.
	\end{theorem}
	\begin{remark}
		Before proving the theorem, we make a remark that Moishezon morphism is stable under base change (which can be viewed as a generation of the Theorem \ref{Moishezon-morphism}), see also \cite{Moishezonmorphism} Remark 17.
	\end{remark}
	
	\begin{proof}
	Let $g: X \rightarrow S$ be a proper, Moishezon morphism. It is bimeromorphic to a projective morphism $X^{\mathrm{p}} \rightarrow S$. We may assume $X^{\mathrm{p}}$ to be normal. Let $Y$ be the normalization of the closure of the graph of $X \dashrightarrow X^{\mathrm{p}}$.
	
	Fix now $s \in S$. Let $Z_s \subset X_s$ be an irreducible component and $W_s \subset Y_s$ an irreducible component that dominates $Z_s$. 
	
	By image of a Moishezon space is Moishezon if the morphism is surjective, it is enough to show that $W_s$ is Moishezon.
	
	If $\pi: Y \rightarrow X^{\mathrm{p}}$ is generically an isomorphism along $W_s$, then $W_s$ is bimeromorphic to an irreducible component of $X_s^{\mathrm{p}}$, hence Moishezon. Otherwise $W_s \subset \operatorname{Ex}(\pi)$. Now $\operatorname{Ex}(\pi) \rightarrow X^{\mathrm{p}}$ is Moishezon by (15) and $\operatorname{dim} \operatorname{Ex}(\pi)<$ $\operatorname{dim} Y=\operatorname{dim} X$. So $W_s$ is contained in a fiber of $\operatorname{Ex}(\pi) \rightarrow S$, hence Moishezon by induction on the dimension.
	\end{proof}
	
	
	\section{Moishezon locus, fiberwise Moishezon morphism is locally Moishezon under smoothness assumption}
	
	We begin this section by defining very big locus, general type locus and Moishezon locus. 
	\begin{definition}[Very big locus, general type locus, Moishezon locus, see \cite{Moishezonmorphism} Definition 18]~\\
	Let $g: X \rightarrow S$ be a proper morphism of normal analytic spaces and $L$ a line bundle on $X$. Set
	\begin{enumerate}
		\item  $\operatorname{VB}_S(L):=\left\{s \in S: L_s\right.$ is very big on $\left.X_s\right\} \subset S$,
		\item $\operatorname{GT}_S(X):=\left\{s \in S: X_s\right.$ is of general type $\} \subset S$,
		\item $\operatorname{MO}_S(X):=\left\{s \in S: X_s\right.$ is Moishezon $\} \subset S$,
		\item $\mathrm{PR}_S(X):=\left\{s \in S: X_s\right.$ is projective $\} \subset S$.
	\end{enumerate}
	here very big means the place $s\in S$ that $$X_s \dashrightarrow \operatorname{Proj}_S(g_* L_s) = (\operatorname{Proj}_S(g_* L))_s$$is birational onto its closure of the image. 
	\end{definition}
	We first show that very big locus admits alternating property i.e. it's either nowhere dense or contains some dense open subset
	
	\begin{theorem}[Alternating property for very big locus, see \cite{Moishezonmorphism} Lemma 19]\label{verybig}~\\
		 Let $g: X \rightarrow S$ be a proper morphism of normal irreducible analytic spaces(and therefore $S$ is integral)  and $L$ a line bundle on $X$. Then $\operatorname{VB}_S(L) \subset S$ is
		(1) either nowhere dense (in the analytic Zariski topology),
		(2) or it contains a dense open subset of $S$, and $g: X \rightarrow S$ is Moishezon.
	\end{theorem}
	\begin{proof}
	By passing to an open subset of $S$, we may assume that $g$ is flat, $g_* L$ is locally free and commutes with restriction to fibers. We get a meromorphic map $\phi: X \dashrightarrow \mathbb{P}_S\left(g_* L\right)$. There is thus a smooth, bimeromorphic model $\pi: X^{\prime} \rightarrow X$ such that $\phi \circ \pi: X^{\prime} \rightarrow \mathbb{P}_S\left(g_* L\right)$ is a morphism.
	
	After replacing $X$ by $X^{\prime}$ and again passing to an open subset of $S$, we may assume that $g$ is flat, $g_* L$ is locally free, commutes with restriction to fibers, and $\phi: X \rightarrow \mathbb{P}_S\left(g_* L\right)$ is a morphism. Let $Y \subset \mathbb{P}_S\left(g_* L\right)$ denote its image and $W \subset X$ the Zariski closed set of points where $\pi: X \rightarrow Y$ is not smooth. Set $Y^{\circ}:=Y \backslash \phi(W)$ and $X^{\circ}:=X \backslash \phi^{-1}(\phi(W))$. The restriction $\phi^{\circ}: X^{\circ} \rightarrow Y^{\circ}$ is then smooth and proper.
	
	We assume that $\phi^{-1}(y)$ is a single point for a dense set in $Y$, hence for a dense set in $Y^{\circ}$. Since $\phi^{\circ}$ is smooth and proper, it is then an isomorphism. Thus $\phi$ is bimeromorphic on every irreducible fiber that has a nonempty intersection with $X^{\circ}$.
	\end{proof}
	As a direct consequence (combined with the classical result by Hacon and Mckernan \cite{HaconMckernan}) we have the general type locus also admits alternating property.
	
	\begin{theorem}[Alternating property for general type locus, see \cite{Moishezonmorphism} Corollary 20]~\\
		Let $g: X \rightarrow S$ be a proper morphism of normal, irreducible analytic spaces. Then $$\text{GT}_S(X) = \{s\in S \mid X_s \text{ is of general type}\}$$(1) either nowhere dense (in the analytic Zariski topology),
		(2) or it contains a dense open subset of $S$, and $g: X \rightarrow S$ is Moishezon
	\end{theorem}
	
	\begin{proof}
		Proof. Using resolution of singularities, we may assume that $X$ is smooth. By passing to an open subset of $S$, we may also assume that $\underline{S}$ and $g$ are smooth. By \cite{HaconMckernan} there is an $m$ (depending only on $\operatorname{dim} X_s$ ) such that $\left|m K_{X_s}\right|$ is very big whenever $X_s$ is of general type. Thus Lemma \ref{verybig} applies to $L=m K_X$.
	\end{proof}
	
	Before proving Theorem 21. Let us first recall the basic idea that being used in \cite{RaoTsai}
	\begin{theorem}[Uncoutnable many fibers are Moishezon with deformation invariance of Hodge number implies the morphism is Moishezon, see \cite{RaoTsai} Proposition 3.15]~\\
		
	    Let $\pi: \mathcal{X} \rightarrow \Delta$ be a one-parameter degeneration. 
		(1) Assume that there exists an uncountable subset $B$ of $\Delta$ such that for each $t \in B$, the fiber $X_t$ admits a line bundle $L_t$ with the property that $c_1\left(L_t\right)$ comes from the restriction to $X_t$ of some cohomology class in $H^2(\mathcal{X}, \mathbb{Z})$.
		(2) Assume further that the Hodge number $h^{0,2}\left(X_t\right):=h^1\left(X_t, \mathcal{O}_{X_t}\right)$ is independent of $t \in \Delta$ (the original theorem require only Hodge (0,1) deformation invariance)
	\end{theorem}
	\begin{proof}
	Apply the sheaf exponential exact sequence so that 
	
	\begin{center}
		\begin{tikzcd} {} & {H^1(\mathcal{X},\mathcal{O}_{\mathcal{X}}^*)} & {H^2(\mathcal{X},\mathbb{Z})} & {H^2(\mathcal{X},\mathcal{O}_X)} & {} \\ {} & {H^1(X_s,\mathcal{O}_{X_s}^*)} & {H^2(X_s,\mathbb{Z})} & {H^2(X_s,\mathcal{O}_{X_s})} & {} \arrow[from=1-1, to=1-2] \arrow[from=1-2, to=1-3] \arrow[from=1-2, to=2-2] \arrow["{e_2}", from=1-3, to=1-4] \arrow[from=1-3, to=2-3] \arrow[from=1-4, to=1-5] \arrow[from=1-4, to=2-4] \arrow[from=2-1, to=2-2] \arrow[from=2-2, to=2-3] \arrow["{e_2}"', from=2-3, to=2-4] \arrow[from=2-4, to=2-5] \end{tikzcd}
	\end{center}
	
	Observe that $$H^2(\mathcal{X},\mathcal{O}_{\mathcal{X}}) \cong R^2\pi_* \mathcal{O}_X(\Delta),\ H^2(X_s, \mathcal{O}_{X_s}) \cong R^2 \pi_* \mathcal{O}_{\mathcal{X}}(s)$$
	
	Indeed
	
	(1) By Cartan B. we have $$H^p(S,R^q \pi_* \mathcal{O}_X) = 0 ,\ p>0$$the Leray spectral sequence arguement thus implies the first isomorphism,
	(2) Since we assume the cohomological constant of $h^{0,2}$, by Grauert base change theorem it will imply the second isomorphism.
	
	Thus the commutative diagram becomes 
	\begin{center}
		\begin{tikzcd}
			&&& {H^0(\mathcal{X},R^2\pi_*\mathcal{O}_{\mathcal{X}})} \\
			{} & {H^1(\mathcal{X},\mathcal{O}_{\mathcal{X}}^*)} & {H^2(\mathcal{X},\mathbb{Z})} & {H^2(\mathcal{X},\mathcal{O}_{\mathcal{X}})} & {} \\
			{} & {H^1(X_s,\mathcal{O}_{X_s}^*)} & {H^2(X_s,\mathbb{Z})} & {H^2(X_s,\mathcal{O}_{X_s})} & {} \\
			&&& {R^2\pi_*\mathcal{O}_{\mathcal{X}}(s)}
			\arrow["\cong", from=1-4, to=2-4]
			\arrow[from=2-1, to=2-2]
			\arrow[from=2-2, to=2-3]
			\arrow[from=2-2, to=3-2]
			\arrow[from=2-3, to=2-4]
			\arrow[from=2-3, to=3-3]
			\arrow[from=2-4, to=2-5]
			\arrow[from=2-4, to=3-4]
			\arrow[from=3-1, to=3-2]
			\arrow[from=3-2, to=3-3]
			\arrow[from=3-3, to=3-4]
			\arrow[from=3-4, to=3-5]
			\arrow["\cong", from=3-4, to=4-4]
		\end{tikzcd}
	\end{center}
	Where we have the evaluation $\text{ev}_s:H^0(\mathcal{X},R^2\pi_* \mathcal{O}_{\mathcal{X}})\to R^2\pi_* \mathcal{O}_{\mathcal{X}}(s)$ in the diagram above.
	
	The idea is to find a cohomology class $c \in H^2(X_s,\mathbb{Z})$ by the simply connectness of $\Delta$ it will lift it to $c\in H^2(\mathcal{X},\mathbb{Z})$, if we can prove the vanishing of $e_2(c) \in H^2(\mathcal{X},\mathcal{O}_{\mathcal{X}})$ then by the exactness of the sequence we can find some global line bundle $\mathcal{L}\in \operatorname{Pic}(\mathcal{X})$.
	
	Observe that the cohomology group $H^2(X_s,\mathbb{Z})\cong H^2(\mathcal{X},\mathbb{Z})$  by Ehresmann's theorem, and this $\mathbb{Z}$ coefficient cohomology group can only have counable many elements, taking uncountble many $L_t$, it must have some $c\in H$ such that there is uncountable many $t$ such that $c_1(L_t) = c$.
	
	Since this $c\in H^2(X_s,\mathbb{Z})$ coming from $\operatorname{Pic}(X_s)$, we have $e_2(c) = 0\in R^2\pi_* \mathcal{O}_{\mathcal{X}}(s)$ and thus if we lift it to $c \in H^2(\mathcal{X},\mathbb{Z})$ the global section $e_2(c) \in H^0(\mathcal{X},R^2 \pi_* \mathcal{O}_{\mathcal{X}})$ it will be zero on uncounable many points. Thus by the identity principle easy to see $e_2(c) =0 \in H^2(\mathcal{X},R^2\pi_* \mathcal{O}_{\mathcal{X}})$. Thus thus lift to some global line bundle $\mathcal{L}\in \operatorname{Pic}(\mathcal{X})$ with the restriction $c_1(\mathcal{L}|_{X_s}) =c_1(L_s)$ and we can now apply the lemma about deformation density of Iitaka-Kodaira dimension below and conclude that $\mathcal{L}$ is indeed a big line bundle to finish the proof.
	
	\end{proof}
	
	\begin{theorem}[Deformation density of Iitaka-Kodaira dimension, see \cite{RaoTsai, LiebermanSernesi}]~\\
		Let $\pi: \mathcal{X} \rightarrow Y$ be a flat family from a complex manifold over a one-dimensional connected complex manifold $Y$ with possibly reducible fibers. 
		
		If there exists a holomorphic line bundle $L$ on $\mathcal{X}$ such that the Kodaira-Iitaka dimension $\kappa\left(L_t\right)=\kappa$ for each $t$ in an uncountable set $B$ of $Y$, then any fiber $X_t$ in $\pi$ has at least one irreducible component $C_t$ with $\kappa\left(\left.L\right|_{C_t}\right) \geq \kappa$. 
		
		In particular, if any fiber $X_t$ for $t \in Y$ is irreducible, then $\kappa\left(L_t\right) \geq \kappa$. 
	\end{theorem}
	\printbibliography	
	
\end{document}
