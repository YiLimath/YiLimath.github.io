\documentclass[11pt]{article}
\usepackage[dvipsnames]{xcolor}
\usepackage{latexsym}
\usepackage{amsmath}
\usepackage{mathrsfs}
\usepackage{tikz-cd}
\usepackage{amssymb}
\usepackage{amsthm}
\usepackage{epsfig}
\usepackage{graphicx}
\usepackage{float}
\newcommand{\handout}[5]{
	\noindent
	\begin{center}
		\framebox{
			\vbox{
				\hbox to 5.78in { {\bf Projectivity Criterira Reading Seminars} \hfill #2 }
				\vspace{4mm}
				\hbox to 5.78in { {\Large \hfill #5  \hfill} }
				\vspace{2mm}
				\hbox to 5.78in { {\em #3 \hfill #4} }
			}
		}
	\end{center}
	\vspace*{4mm}
}

\newcommand{\lecture}[4]{\handout{#1}{#2}{#3}{#4}{Note #1 (draft version 0)}}
\usepackage{amsthm}

\theoremstyle{definition}
\newtheorem{theorem}{Theorem}[section]
\newtheorem{corollary}[theorem]{Corollary}
\newtheorem{lemma}[theorem]{Lemma}
\newtheorem{observation}[theorem]{Observation}
\newtheorem{proposition}[theorem]{Proposition}
\newtheorem{definition}[theorem]{Definition}
\newtheorem{claim}[theorem]{Claim}
\newtheorem{remark}[theorem]{Remark}
\newtheorem{fact}[theorem]{Fact}
\newtheorem{assumption}[theorem]{Assumption}

% 1-inch margins, from fullpage.sty by H.Partl, Version 2, Dec. 15, 1988.
\topmargin 0pt
\advance \topmargin by -\headheight
\advance \topmargin by -\headsep
\textheight 8.9in
\oddsidemargin 0pt
\evensidemargin \oddsidemargin
\marginparwidth 0.5in
\textwidth 6.5in

\parindent 0in
\parskip 1.5ex
%\renewcommand{\baselinestretch}{1.25}

\usepackage[backend=bibtex,style=alphabetic,maxalphanames=4,minalphanames=4,sorting=nty]{biblatex}
\addbibresource{mybib.bib}


\usepackage{fancyhdr}
\pagestyle{fancy}
\lhead{Birational Geometry Reading Seminars}
\rhead{\thepage}
%\cfoot{center of the footer!}
\renewcommand{\headrulewidth}{0.4pt}
\renewcommand{\footrulewidth}{0.4pt}

\usepackage{hyperref}

\hypersetup{
	colorlinks=true,
	linkcolor=Blue,
	filecolor=YellowOrange,
	citecolor = WildStrawberry,      
	urlcolor=cyan,
}


\begin{document}
	
	\lecture{3 --- 18, 09, 2024}{Fall 2024}{}{Yi Li}
	\section{Overview}
	
	This time we will finish the proof of the main theorem of the paper \cite{Seshardri}, Theorem 2. Before proving this openess of projectivity theorem, we need another result about alternating property of projective locus of a proper (Moishezon) morphism. Combined with the theorem about projectivity of very general fibers that we proved last time, the proof of the main theorem is immediate.
	
	\section{Alternating property of projective locus}
	In this section, we will focus on the proof of the following theorem
	
	\begin{theorem}[Alternating property of projective locus, see \cite{Seshardri}, Proposition 17]~
	
	Let $g: X \rightarrow S$ be a proper morphism of normal, irreducible analytic spaces. Then there is a dense, Zariski open subset $S^{\circ} \subset S$ such that
	
	(1) either $X$ is locally projective over $S^{\circ}$,
	
	(2) or $\mathrm{PR}_S(X) \cap S^{\circ}$ is locally contained in a countable union of Zariski closed, nowhere dense subsets.	
	\end{theorem}
	
	\begin{proof}
	The main technical tool is the sheaf exponential sequence with its restriction on the fiber. 
	
	
	\end{proof}
	\section{Proof of the main theorem}
	
	\begin{theorem}[Openess of projectivity, see \cite{Seshardri}, Theorem 2]~
		
		Let $g: X \rightarrow S$ be a proper morphism of complex analytic spaces and $S^* \subset S$ a dense, Zariski open subset such that $g$ is flat over $S^*$. Assume that
		
		(1) $X_0$ is projective for some $0 \in S$,
		
		(2) the fibers $X_s$ have rational singularities for $s \in S^*$, and
		
		(3) $g$ is bimeromorphic to a projective morphism $g^{\mathrm{p}}: X^{\mathrm{p}} \rightarrow S$.
		
		Then there is a Zariski open neighborhood $0 \in U \subset S$ and a locally closed, Zariski stratification $U \cap S^*=\cup_i S_i$ such that each
		
		$$
		\left.g\right|_{X_i}: X_i:=g^{-1}\left(S_i\right) \rightarrow S_i \text { is projective. }
		$$
		
	\end{theorem}
	
	\begin{proof}
	
	\end{proof}
	
%	
%	\section{The fiber of the Moishezon morphism is again Moishezon}
%	
%	\begin{theorem}[The fiber of the Moishezon morphism is again Moishezon, see \cite{Moishezonmorphism}, Corollary 16]\label{Moishezon-morphism}
%	
%	\end{theorem}
%	
%	
%	\begin{remark}
%	\cite{Moishezonmorphism} 
%	\end{remark}
%	
%	\begin{proof}
%		
%	\end{proof}
%
%	\begin{center}
%		\begin{tikzcd}
%			&&& {H^0(\mathcal{X},R^2\pi_*\mathcal{O}_{\mathcal{X}})} \\
%			{} & {H^1(\mathcal{X},\mathcal{O}_{\mathcal{X}}^*)} & {H^2(\mathcal{X},\mathbb{Z})} & {H^2(\mathcal{X},\mathcal{O}_{\mathcal{X}})} & {} \\
%			{} & {H^1(X_s,\mathcal{O}_{X_s}^*)} & {H^2(X_s,\mathbb{Z})} & {H^2(X_s,\mathcal{O}_{X_s})} & {} \\
%			&&& {R^2\pi_*\mathcal{O}_{\mathcal{X}}(s)}
%			\arrow["\cong", from=1-4, to=2-4]
%			\arrow[from=2-1, to=2-2]
%			\arrow[from=2-2, to=2-3]
%			\arrow[from=2-2, to=3-2]
%			\arrow[from=2-3, to=2-4]
%			\arrow[from=2-3, to=3-3]
%			\arrow[from=2-4, to=2-5]
%			\arrow[from=2-4, to=3-4]
%			\arrow[from=3-1, to=3-2]
%			\arrow[from=3-2, to=3-3]
%			\arrow[from=3-3, to=3-4]
%			\arrow[from=3-4, to=3-5]
%			\arrow["\cong", from=3-4, to=4-4]
%		\end{tikzcd}
%	\end{center}
%
%	\begin{figure}[H]
%		\centering
%		\includegraphics[width=0.85\linewidth]{"Moishezon morphism definition"}
%		\caption{Comparison between different definitions for Moishezon morphism}
%		\label{fig:moishezon-morphism-definition}
%	\end{figure}
%	
	\printbibliography	
	
\end{document}
