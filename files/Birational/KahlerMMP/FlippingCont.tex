\documentclass[11pt]{article}
\usepackage[dvipsnames]{xcolor}
\usepackage{latexsym}
\usepackage{amsmath}
\usepackage{mathrsfs}
\usepackage{tikz-cd}
\usepackage{amssymb}
\usepackage{amsthm}
\usepackage{epsfig}
\usepackage{graphicx}
\usepackage{float}
\newcommand{\handout}[5]{
	\noindent
	\begin{center}
		\framebox{
			\vbox{
				\hbox to 5.78in { {\bf Divisorial contraction in K\"ahler MMP reading notes} \hfill #2 }
				\vspace{4mm}
				\hbox to 5.78in { {\Large \hfill #5  \hfill} }
				\vspace{2mm}
				\hbox to 5.78in { {\em #3 \hfill #4} }
			}
		}
	\end{center}
	\vspace*{4mm}
}

\newcommand{\lecture}[4]{\handout{#1}{#2}{#3}{#4}{Lecture #1 (draft version)}}
\usepackage{amsthm}

\theoremstyle{definition}
\newtheorem{theorem}{Theorem}
\newtheorem{corollary}[theorem]{Corollary}
\newtheorem{lemma}[theorem]{Lemma}
\newtheorem{observation}[theorem]{Observation}
\newtheorem{proposition}[theorem]{Proposition}
\newtheorem{definition}[theorem]{Definition}
\newtheorem{claim}[theorem]{Claim}
\newtheorem{remark}[theorem]{Remark}
\newtheorem{fact}[theorem]{Fact}
\newtheorem{assumption}[theorem]{Assumption}

% 1-inch margins, from fullpage.sty by H.Partl, Version 2, Dec. 15, 1988.
\topmargin 0pt
\advance \topmargin by -\headheight
\advance \topmargin by -\headsep
\textheight 8.9in
\oddsidemargin 0pt
\evensidemargin \oddsidemargin
\marginparwidth 0.5in
\textwidth 6.5in

\parindent 0in
\parskip 1.5ex
%\renewcommand{\baselinestretch}{1.25}

%\usepackage{biblatex}


\usepackage{fancyhdr}
\pagestyle{fancy}
\lhead{K\"ahler MMP reading seminars}
\rhead{\thepage}
%\cfoot{center of the footer!}
\renewcommand{\headrulewidth}{0.4pt}
\renewcommand{\footrulewidth}{0.4pt}

\usepackage{hyperref}

\hypersetup{
	colorlinks=true,
	linkcolor=Blue,
	filecolor=YellowOrange,
	citecolor = WildStrawberry,      
	urlcolor=cyan,
}


\begin{document}
	
	\lecture{4 --- 25, 02, 2025}{Spring 2025}{}{Yi Li}
	\tableofcontents
	\section{Overview}
	
	\section{Das-Hacon's approach to divisorial contraction for K\"ahler 3-fold MMP}
	In this section, we will prove the following theorem.
	\begin{theorem}[{\cite[Theorem 6.9]{DH24}}]\label{DHdivisorial}
		Let $(X,B)$ be a strong $\mathbb{Q}$-factorial K\"ahler 3-fold KLT pair. With the following condition holds
		\begin{enumerate}
			\item $K_X+B $ is pseudo-effective
			\item $\alpha = [K_X+B + \beta]$ is nef and big class such that $\beta$ is K\"ahler,
			\item The negative extremal ray $R = \overline{\text{NA}}(X) \cap \alpha^\perp$ is divisorial.
		\end{enumerate}
		Then there exists an $\alpha$-trivial divisorial contraction $$f:X\to Z,$$such that there exist some K\"ahler form $\alpha_Z$ on $Z$ such that $\phi^* (\alpha_Z) = \alpha$.
	\end{theorem}
	Before going to the proof let us briefly sketch the idea. We first try to prove that the null locus $\text{Null}(\alpha)$ is the Moishezon surface whose smooth model is projective uniruled. We then take a DLT modification $$\varphi:(X',\Delta')\to (X,\Delta)$$ of the pair $(X,\Delta = B+(1-b)S)$ (note that this pair $(X,\Delta)$ differs from the original pair $(X,B)$ and it is not a KLT pair). 
	
	We show that the DLT modification $\varphi$ preserve the geometry outside the null locus $\text{Null}(\alpha)$. We then run the relative K\"ahler MMP for $(X',\Delta')$ over $(X,\Delta)$, which becomes the core of the proof. Since it's K\"ahler 3-fold MMP, the termination is known. So that it's possible to produce positivity (say $K_{X^m} + \Delta^m$ is nef over $(X,\Delta)$) by the termination theorem. 
	
	We need to control the divisors being contracted in the MMP. 
	
	
	So that the induced bimeromorphic map $f:X\dashrightarrow X^m$ is a morphism, and this is the divisorial contraction we want. 
	
	In the final step, we will show that the base point freeness holds for the divisorial contraction, say $\alpha$ as pull back of some K\"ahler form $\alpha_Z$ down stairs. 
	
	\subsection{The null locus is a Moishezon surface whose smooth model is projective uniruled}
	In this subsection, we will proof the following lemma.
	\begin{lemma}
		In the same setting as Theorem \ref{DHdivisorial}. The null locus $\text{Null}(\alpha)$ is a irreducible Moishezon surface, whose smooth model is projective uniruled. Such that the curves in the negative extremal ray $R$ covers the surface $S$ with $$R \cdot S <0.$$
	\end{lemma}
	\begin{remark}
		Let us breifly sketch the idea. The class $\alpha|_S, \alpha|_{S^\nu}, \alpha|_{S'}$ play important role in this lemma (for simplicity let us assume for now that $S$ is a smooth surface). The idea is to try to use that if a smooth surface is not pseudo-effective, then it's uniruled projective surface. The non-pseudo-effectiveness comes from some intersection number analysis. To be more precise, we will use that $S = \text{Null}(\alpha)$, so that volume $\text{vol}(\alpha|_S)  = (\alpha|_S)^2= 0$ (by definition of null locus). In particular, the restriction $\alpha|_S$ can not be a big class. On the other hand, we can apply adjunction to $$\alpha|_S  = (K_X+B+\beta)|_S.$$
		If the coefficient of $S$ in $B$ is 1, then everything is nice and we get $$\alpha|_S = (K_X+B ' + S + \beta)|_S = K_S+ B'|_S + \beta|_S.$$Since $B'|_S\ge 0$ and $\beta|_S$ Kahler, this will imply that $K_S$ can not be pseudo-effective. 
		
		However, the coefficient of $S$ in $B$ is not 1, so that we need to take some scaling, 
		
		What nice on the projective uniruled surface is that the (0,2)-Hodge number is 0, so that the Bott-Chern class can be realized as a $\mathbf{R}$-divisor (which is also a $\mathbb{R}$-curve on the surface). 
		
		Finally, we need to prove that $R\cdot S<0$. To do this, Batyrev cone theorem for mobile curve is applied. 
	\end{remark}
	\begin{proof}
		
	\end{proof}
	
	\subsection{Take DLT modification}
	
	\subsection{Run the relative MMP}
	
	\subsection{Control the set of divisors being contracted}
	
	\subsection{Proof of the base point freeness}
	
	
	\section{H\"oring-Peternell's approach for K\"ahler 3-fold MMP}
	
	
	
	%	\begin{center}
		%		\begin{tikzcd}
			%			&&& {H^0(\mathcal{X},R^2\pi_*\mathcal{O}_{\mathcal{X}})} \\
			%			{} & {H^1(\mathcal{X},\mathcal{O}_{\mathcal{X}}^*)} & {H^2(\mathcal{X},\mathbb{Z})} & {H^2(\mathcal{X},\mathcal{O}_{\mathcal{X}})} & {} \\
			%			{} & {H^1(X_s,\mathcal{O}_{X_s}^*)} & {H^2(X_s,\mathbb{Z})} & {H^2(X_s,\mathcal{O}_{X_s})} & {} \\
			%			&&& {R^2\pi_*\mathcal{O}_{\mathcal{X}}(s)}
			%			\arrow["\cong", from=1-4, to=2-4]
			%			\arrow[from=2-1, to=2-2]
			%			\arrow[from=2-2, to=2-3]
			%			\arrow[from=2-2, to=3-2]
			%			\arrow[from=2-3, to=2-4]
			%			\arrow[from=2-3, to=3-3]
			%			\arrow[from=2-4, to=2-5]
			%			\arrow[from=2-4, to=3-4]
			%			\arrow[from=3-1, to=3-2]
			%			\arrow[from=3-2, to=3-3]
			%			\arrow[from=3-3, to=3-4]
			%			\arrow[from=3-4, to=3-5]
			%			\arrow["\cong", from=3-4, to=4-4]
			%		\end{tikzcd}
		%	\end{center}
	
	%	\begin{figure}[H]
		%		\centering
		%		\includegraphics[width=0.85\linewidth]{"Moishezon morphism definition"}
		%		\caption{Comparison between different definitions for Moishezon morphism}
		%		\label{fig:moishezon-morphism-definition}
		%	\end{figure}
	\bibliographystyle{amsalpha}
	\bibliography{mybib.bib}
\end{document}
