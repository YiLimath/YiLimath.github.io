\documentclass[11pt]{article}
\usepackage[dvipsnames]{xcolor}
\usepackage{latexsym}
\usepackage{amsmath}
\usepackage{mathrsfs}
\usepackage{tikz-cd}
\usepackage{amssymb}
\usepackage{amsthm}
\usepackage{epsfig}
\usepackage{graphicx}
\usepackage{float}
\newcommand{\handout}[5]{
	\noindent
	\begin{center}
		\framebox{
			\vbox{
				\hbox to 5.78in { {\bf Flipping contraction in K\"ahler MMP reading notes} \hfill #2 }
				\vspace{4mm}
				\hbox to 5.78in { {\Large \hfill #5  \hfill} }
				\vspace{2mm}
				\hbox to 5.78in { {\em #3 \hfill #4} }
			}
		}
	\end{center}
	\vspace*{4mm}
}

\newcommand{\lecture}[4]{\handout{#1}{#2}{#3}{#4}{Lecture #1 (draft version)}}
\usepackage{amsthm}

\theoremstyle{definition}
\newtheorem{theorem}{Theorem}
\newtheorem{corollary}[theorem]{Corollary}
\newtheorem{lemma}[theorem]{Lemma}
\newtheorem{observation}[theorem]{Observation}
\newtheorem{proposition}[theorem]{Proposition}
\newtheorem{definition}[theorem]{Definition}
\newtheorem{claim}[theorem]{Claim}
\newtheorem{remark}[theorem]{Remark}
\newtheorem{fact}[theorem]{Fact}
\newtheorem{assumption}[theorem]{Assumption}

% 1-inch margins, from fullpage.sty by H.Partl, Version 2, Dec. 15, 1988.
\topmargin 0pt
\advance \topmargin by -\headheight
\advance \topmargin by -\headsep
\textheight 8.9in
\oddsidemargin 0pt
\evensidemargin \oddsidemargin
\marginparwidth 0.5in
\textwidth 6.5in

\parindent 0in
\parskip 1.5ex
%\renewcommand{\baselinestretch}{1.25}

%\usepackage{biblatex}


\usepackage{fancyhdr}
\pagestyle{fancy}
\lhead{K\"ahler MMP reading seminars}
\rhead{\thepage}
%\cfoot{center of the footer!}
\renewcommand{\headrulewidth}{0.4pt}
\renewcommand{\footrulewidth}{0.4pt}

\usepackage{hyperref}

\hypersetup{
	colorlinks=true,
	linkcolor=Blue,
	filecolor=YellowOrange,
	citecolor = WildStrawberry,      
	urlcolor=cyan,
}


\begin{document}
	
	\lecture{4 --- 25, 02, 2025}{Spring 2025}{}{Yi Li}
	\tableofcontents
	\section{Overview}
	
	\section{Das-Hacon's approach to flipping contraction for K\"ahler 3-fold MMP}
	In this section, we will prove the following theorem.
	\begin{theorem}[{\cite[Theorem 6.9]{DH24}}]\label{DHdivisorial}
		Let $(X,B)$ be a strong $\mathbb{Q}$-factorial K\"ahler 3-fold KLT pair. With the following condition holds
		\begin{enumerate}
			\item $K_X+B $ is pseudo-effective
			\item $\alpha = [K_X+B + \beta]$ is nef and big class such that $\beta$ is K\"ahler,
			\item The negative extremal ray $R = \overline{\text{NA}}(X) \cap \alpha^\perp$ is flipping type.
		\end{enumerate}
		Then there exists an $\alpha$-trivial flipping contraction $$f:X\to Z,$$such that there exist some K\"ahler form $\alpha_Z$ on $Z$ such that $\phi^* (\alpha_Z) = \alpha$. And the flip exist.
	\end{theorem}
	\begin{remark}
		Before proving the theorem, let us briefly sketch the idea of the proof. 
	\end{remark}
	%\subsection{The null locus is a Moishezon surface whose smooth model is projective uniruled}
	\subsection{Apply the DLT modification}
	\subsection{Find a negative extremal ray on one of the component $S' \subset \lfloor{\Delta'}\rfloor$}
	\subsection{Apply the PLT contraction theorem}
	In this step, we try to extend to contraction from $S'\to T'$ to $X'\to Z'$. 
	
	\subsection{Prove the contraction morphism $X'\to Z'$ is divisorial or flipping contraction}
	We try to prove the contraction we just contructed is a divisorial or flipping contraction, thus it's an MMP step.
	
	\subsection{Control the MMP}
	Similar to the divisorial contraction case, the MMP we just contructed has some good properties, we will show that 
	\subsection{Find the contraction $f:X\to Z$}
	In this step, we take the normalization on the graph of the map $\psi:X\dashrightarrow X^+$. Let $\eta:=\alpha+\delta\left(K_X+B\right)$, we try to show that $$E = p^*\eta - q^* \eta^+,$$satisfies the condition that $-E|_E$ is ample. So that we can apply the Grauert contraction theorem, and get a contraction $g:W\to Z$. 
	
	We then try to show that this contraction will induce $f:X\to Z$ and $f^+ : X^+ \to Z$ as the diagram below shows. 
	\begin{center}
		\begin{tikzcd}[ampersand replacement=\&] {X'} \&\& {X^+} \\ X \&\& W \\ \&\&\& Z \arrow["\phi", dashed, from=1-1, to=1-3] \arrow["\nu"', from=1-1, to=2-1] \arrow["{f^+}", from=1-3, to=3-4] \arrow["\psi"{description}, dashed, from=2-1, to=1-3] \arrow["f"', from=2-1, to=3-4] \arrow["q", from=2-3, to=1-3] \arrow["p"{description}, from=2-3, to=2-1] \arrow["g"{description}, from=2-3, to=3-4] \end{tikzcd}
	\end{center}
	
	We then show that $f:X\to Z$ will contract every curves in the negative extremal ray $R$ and it's a $(K_X+B)$-negative contraction.
	
	
	\subsection{Prove base point freeness}
	Just as what we did in the divisorial contraction case, we try to use that fact that proper bimeromorphic morphism $f:X\to Z$ between K\"ahler varieties with rational singularity satisfies the condition $$\text{im}(f^*) = \{\alpha \in H^{1,1}_{\rm{BC}}(X)\mid \alpha \cdot C = 0 ,\ \forall\ C\in N_1(X/Z)\}.$$Then apply singular version Demailly-P\u{a}un Kahlerness criterion to the big and nef class $\alpha_Z$.
	
	
	\subsection{Prove the existence of flips in any dimension}
	
	
	\section{H\"oring-Peternell's approach for flipping contraction}
	
	
	
	%	\begin{center}
		%		\begin{tikzcd}
			%			&&& {H^0(\mathcal{X},R^2\pi_*\mathcal{O}_{\mathcal{X}})} \\
			%			{} & {H^1(\mathcal{X},\mathcal{O}_{\mathcal{X}}^*)} & {H^2(\mathcal{X},\mathbb{Z})} & {H^2(\mathcal{X},\mathcal{O}_{\mathcal{X}})} & {} \\
			%			{} & {H^1(X_s,\mathcal{O}_{X_s}^*)} & {H^2(X_s,\mathbb{Z})} & {H^2(X_s,\mathcal{O}_{X_s})} & {} \\
			%			&&& {R^2\pi_*\mathcal{O}_{\mathcal{X}}(s)}
			%			\arrow["\cong", from=1-4, to=2-4]
			%			\arrow[from=2-1, to=2-2]
			%			\arrow[from=2-2, to=2-3]
			%			\arrow[from=2-2, to=3-2]
			%			\arrow[from=2-3, to=2-4]
			%			\arrow[from=2-3, to=3-3]
			%			\arrow[from=2-4, to=2-5]
			%			\arrow[from=2-4, to=3-4]
			%			\arrow[from=3-1, to=3-2]
			%			\arrow[from=3-2, to=3-3]
			%			\arrow[from=3-3, to=3-4]
			%			\arrow[from=3-4, to=3-5]
			%			\arrow["\cong", from=3-4, to=4-4]
			%		\end{tikzcd}
		%	\end{center}
	
	%	\begin{figure}[H]
		%		\centering
		%		\includegraphics[width=0.85\linewidth]{"Moishezon morphism definition"}
		%		\caption{Comparison between different definitions for Moishezon morphism}
		%		\label{fig:moishezon-morphism-definition}
		%	\end{figure}
	\bibliographystyle{amsalpha}
	\bibliography{mybib.bib}
\end{document}
