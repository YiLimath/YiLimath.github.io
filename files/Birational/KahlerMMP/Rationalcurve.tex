\documentclass[11pt]{article}
\usepackage[dvipsnames]{xcolor}
\usepackage{latexsym}
\usepackage{amsmath}
\usepackage{mathrsfs}
\usepackage{tikz-cd}
\usepackage{amssymb}
\usepackage{amsthm}
\usepackage{epsfig}
\usepackage{graphicx}
\usepackage{float}
\newcommand{\handout}[5]{
	\noindent
	\begin{center}
		\framebox{
			\vbox{
				\hbox to 5.78in { {\bf Rational curves on analytic space reading note} \hfill #2 }
				\vspace{4mm}
				\hbox to 5.78in { {\Large \hfill #5  \hfill} }
				\vspace{2mm}
				\hbox to 5.78in { {\em #3 \hfill #4} }
			}
		}
	\end{center}
	\vspace*{4mm}
}

\newcommand{\lecture}[4]{\handout{#1}{#2}{#3}{#4}{Lecture #1 (draft version)}}
\usepackage{amsthm}

\theoremstyle{definition}
\newtheorem{theorem}{Theorem}
\newtheorem{corollary}[theorem]{Corollary}
\newtheorem{lemma}[theorem]{Lemma}
\newtheorem{observation}[theorem]{Observation}
\newtheorem{proposition}[theorem]{Proposition}
\newtheorem{definition}[theorem]{Definition}
\newtheorem{claim}[theorem]{Claim}
\newtheorem{remark}[theorem]{Remark}
\newtheorem{fact}[theorem]{Fact}
\newtheorem{assumption}[theorem]{Assumption}

% 1-inch margins, from fullpage.sty by H.Partl, Version 2, Dec. 15, 1988.
\topmargin 0pt
\advance \topmargin by -\headheight
\advance \topmargin by -\headsep
\textheight 8.9in
\oddsidemargin 0pt
\evensidemargin \oddsidemargin
\marginparwidth 0.5in
\textwidth 6.5in

\parindent 0in
\parskip 1.5ex
%\renewcommand{\baselinestretch}{1.25}

%\usepackage{biblatex}


\usepackage{fancyhdr}
\pagestyle{fancy}
\lhead{K\"ahler MMP reading seminars}
\rhead{\thepage}
%\cfoot{center of the footer!}
\renewcommand{\headrulewidth}{0.4pt}
\renewcommand{\footrulewidth}{0.4pt}

\usepackage{hyperref}

\hypersetup{
	colorlinks=true,
	linkcolor=Blue,
	filecolor=YellowOrange,
	citecolor = WildStrawberry,      
	urlcolor=cyan,
}


\begin{document}
	
	\lecture{9 --- 02, 18, 2025}{Spring 2025}{}{Yi Li}
	\section{Overview}
	In this note, we will study the rational curves on Moishezon space and K\"ahler space. The major references of this note are \cite{CH20} and \cite{HP16}.
	\section{H\"oring-Peternell's approach of Mori bend and break for K\"ahler 3-fold}
	In this section, we will briefly summarize the idea of H\"oring-Peternell on Mori bend and break for K\"ahler 3-fold \cite{HP16}.
	\begin{theorem}
		Let $X$ be a normal $\mathbb{Q}$-factorial compact K\"ahler 3-fold with terminal singularity, with $K_X$ being pseudo-effective.
		\begin{enumerate}
			\item If $C$ is a curve with large anti-canonical degree, then it can break into $$[C] = [C_1]+ [C_2]$$
			\item Let $\overline{\text{NE}(X)}$ has a $K_X$-negative extremal ray $\mathbb{R}_+[\Gamma]$, such that the representative $\Gamma$ is not very rigid. Then we can find a representable $C\in \mathbb{R}_+ [\Gamma]$ such that $\dim_C\text{Chow}(X) >0$, and $\mathbb{R}_+[\Gamma]$ contains ratioal curves. \end{enumerate}
		
	\end{theorem}
	\begin{remark}
		Let us first briefly sketch the idea of the proof: The general idea is if the anti-canonical degree is large, then the curve $C$ is deformable i.e. $\dim_C\text{Chow}(X)>0$.
		
		We then prove that the deformation of the curve contains in a component $S_i$ of negative par of Zariski decomposition $N(K_X)$. (Note that the surface $S_i$ has $K_{S_i}$ not pseudo-effective, thus it is a uniruled surface)
		
		Therefore, we reduce the problem onto the surface $S_i$. We try to prove that $K_{S_i}$-negative curve on the uniruled surface breaks and produce a rational curve.
	\end{remark}\begin{proof}
		
	\end{proof}
	
	
	\section{Cao-H\"oring's approach produce rational curve for K\"ahler manifold}
	\subsection{Pseudo effectiveness of the relative adjoint class}
	The major technical tools that will be used in the proof of the Cao-H\"oring's theorem is the following pseudo-effectiveness theorem. 
	
	\begin{theorem}
		
	\end{theorem}
	\begin{remark}
		
	\end{remark}
	
	\subsection{Cao-H\"oring's main theorem}
	Now we can prove the main theorem of the \cite{CH20}.
	
	
	
	
	%	\begin{figure}[H]
		%		\centering
		%		\includegraphics[width=0.85\linewidth]{"Moishezon morphism definition"}
		%		\caption{Comparison between different definitions for Moishezon morphism}
		%		\label{fig:moishezon-morphism-definition}
		%	\end{figure}
	\bibliographystyle{amsalpha}
	\bibliography{mybib.bib}
\end{document}
