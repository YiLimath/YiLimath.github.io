\documentclass[11pt]{article}
\usepackage[dvipsnames]{xcolor}
\usepackage{latexsym}
\usepackage{amsmath}
\usepackage{mathrsfs}
\usepackage{tikz-cd}
\usepackage{amssymb}
\usepackage{amsthm}
\usepackage{epsfig}
\usepackage{graphicx}
\usepackage{float}
\newcommand{\handout}[5]{
	\noindent
	\begin{center}
		\framebox{
			\vbox{
				\hbox to 5.78in { {\bf BDPP reading notes} \hfill #2 }
				\vspace{4mm}
				\hbox to 5.78in { {\Large \hfill #5  \hfill} }
				\vspace{2mm}
				\hbox to 5.78in { {\em #3 \hfill #4} }
			}
		}
	\end{center}
	\vspace*{4mm}
}

\newcommand{\lecture}[4]{\handout{#1}{#2}{#3}{#4}{Lecture #1 (version 0.0)}}
\usepackage{amsthm}

\theoremstyle{definition}
\newtheorem{theorem}{Theorem}
\newtheorem{corollary}[theorem]{Corollary}
\newtheorem{lemma}[theorem]{Lemma}
\newtheorem{observation}[theorem]{Observation}
\newtheorem{proposition}[theorem]{Proposition}
\newtheorem{definition}[theorem]{Definition}
\newtheorem{claim}[theorem]{Claim}
\newtheorem{remark}[theorem]{Remark}
\newtheorem{fact}[theorem]{Fact}
\newtheorem{assumption}[theorem]{Assumption}

% 1-inch margins, from fullpage.sty by H.Partl, Version 2, Dec. 15, 1988.
\topmargin 0pt
\advance \topmargin by -\headheight
\advance \topmargin by -\headsep
\textheight 8.9in
\oddsidemargin 0pt
\evensidemargin \oddsidemargin
\marginparwidth 0.5in
\textwidth 6.5in

\parindent 0in
\parskip 1.5ex
%\renewcommand{\baselinestretch}{1.25}

%\usepackage{biblatex}


\usepackage{fancyhdr}
\pagestyle{fancy}
\lhead{K\"ahler MMP reading seminars}
\rhead{\thepage}
%\cfoot{center of the footer!}
\renewcommand{\headrulewidth}{0.4pt}
\renewcommand{\footrulewidth}{0.4pt}

\usepackage{hyperref}

\hypersetup{
	colorlinks=true,
	linkcolor=Blue,
	filecolor=YellowOrange,
	citecolor = WildStrawberry,      
	urlcolor=cyan,
}


\begin{document}
	
	\lecture{Supplement 3 --- 17, 02, 2025}{Spring 2025}{}{Yi Li}
	\section{Overview}
	This aim of this note is to introduce the BDPP theorem for projective \cite{BDPP} and Kahler manifold \cite{Ou}. Varies applications of the BDPP theorem are shown. 
	
	\section{Transcendentla cone}
	On a compact K\"ahler manifold, there may not have plently of divisors. To make sense varies positivities, it is necessary to introduce the transcendentla cones. 
	\section{Duality between varies cones}
	The following theorem shows the duality between pseudo-effective cone and movable cone on the projective manifold.
	\begin{theorem}
		Let $X$ be a projective manifold, then the pseudo-effective cone is dual to the cone of movable curves $$\mathcal{E} = \overline{\text{Mov}(X)}^\vee.$$In other words, a divisor is pseudo-effective iff it has non-negative intersection with any movable curves.    \end{theorem}
	\begin{remark}
		David \cite{David19} proved ... 
	\end{remark}
	
	\section{Characterization of the projective uniruled manifold}
	
	The projective uniruled manifold is characterized by the pseudo-effectiveness of the canonical bundle.
	\begin{theorem}[{\cite[Corollary 0.3]{BDPP}}]
		Let $X$ be a projective manifold. Then $X$ is uniruled iff $K_X$ is not pseudo-effectiveness.
	\end{theorem}
	\begin{remark}
		One direction of the proof is easy, and can be adopted to the K\"ahler manifold. The converse direction (say $K_X$ is not pseudo-effective) implies uniruled of $X$ is non-trivial, which requires the Mori bend and break technique and the duality between pseudo-effective cone and movable cone.
	\end{remark}
	\begin{remark}
		Miyaoka and Mori \cite{MM86} proved that a projective manifold is uniruled iff there exist an open subset over which there exist a $K_X$-negative curve passing through it. For more discussion about Miyaokao-Mori theorem (and varies properties of uniruled manifold) see my Note 15.
	\end{remark}
	
	
	We can generalize the BDPP theorem to the singular case.
	\begin{theorem}
		Let $(X,B)$ be a $\mathbb{Q}$-factorial log pair. If $K_X+B$ is not pseudo-effective, then $X$ is uniruled.
	\end{theorem}
	\begin{remark}
		Rational curves on singular space is tricky. See more discussion on my notes note-9 Rational curves on Moishezon space, Kaehler varieties.
	\end{remark}
	\begin{proof}
		Taking the log resolution $$f:X'\to X, $$such that $f^*(K_X+B) =  K_{X'}+B'$. Since being uniruled is birational invariant, if $X$ is not uniruled, then so it is $X'$. Then by the BDPP theorem we just proved, $K_{X'}$ is pseudo-effective, thus $K_X$ is pseudo-effective. Since $B$ is effective, $K_X+B$ is pseudo-effective. 
	\end{proof}
	\section{Proof of BDPP conjecture for K\"ahler manifold}
	Recently, \cite{Ou} proved the BDPP conjecture for the compact K\"ahler manifold. In this section, we will briefly introduce the result that he proved.
	
	\subsection{Algebraic integrability criteria under K\"ahler setting}
	\subsection{Proof of BDPP conjecture for compact K\"ahler manifold}
	
	\section{Varies applications}
	\subsection{Applications of duality of pseudo-effective cone and cone of movable curves}
	
	\subsection{Producing rational curves using BDPP conjecture}
	
	\subsection{Cone theorem using BDPP conjecture}
	
	%	\begin{center}
		%		\begin{tikzcd}
			%			&&& {H^0(\mathcal{X},R^2\pi_*\mathcal{O}_{\mathcal{X}})} \\
			%			{} & {H^1(\mathcal{X},\mathcal{O}_{\mathcal{X}}^*)} & {H^2(\mathcal{X},\mathbb{Z})} & {H^2(\mathcal{X},\mathcal{O}_{\mathcal{X}})} & {} \\
			%			{} & {H^1(X_s,\mathcal{O}_{X_s}^*)} & {H^2(X_s,\mathbb{Z})} & {H^2(X_s,\mathcal{O}_{X_s})} & {} \\
			%			&&& {R^2\pi_*\mathcal{O}_{\mathcal{X}}(s)}
			%			\arrow["\cong", from=1-4, to=2-4]
			%			\arrow[from=2-1, to=2-2]
			%			\arrow[from=2-2, to=2-3]
			%			\arrow[from=2-2, to=3-2]
			%			\arrow[from=2-3, to=2-4]
			%			\arrow[from=2-3, to=3-3]
			%			\arrow[from=2-4, to=2-5]
			%			\arrow[from=2-4, to=3-4]
			%			\arrow[from=3-1, to=3-2]
			%			\arrow[from=3-2, to=3-3]
			%			\arrow[from=3-3, to=3-4]
			%			\arrow[from=3-4, to=3-5]
			%			\arrow["\cong", from=3-4, to=4-4]
			%		\end{tikzcd}
		%	\end{center}
	
	%	\begin{figure}[H]
		%		\centering
		%		\includegraphics[width=0.85\linewidth]{"Moishezon morphism definition"}
		%		\caption{Comparison between different definitions for Moishezon morphism}
		%		\label{fig:moishezon-morphism-definition}
		%	\end{figure}
	\bibliographystyle{amsalpha}
	\bibliography{mybib.bib}
\end{document}
