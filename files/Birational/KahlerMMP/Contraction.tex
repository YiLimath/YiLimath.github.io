\documentclass[11pt]{article}
\usepackage[dvipsnames]{xcolor}
\usepackage{latexsym}
\usepackage{amsmath}
\usepackage{mathrsfs}
\usepackage{tikz-cd}
\usepackage{amssymb}
\usepackage{amsthm}
\usepackage{epsfig}
\usepackage{graphicx}
\usepackage{float}
\newcommand{\handout}[5]{
	\noindent
	\begin{center}
		\framebox{
			\vbox{
				\hbox to 5.78in { {\bf Contraction in K\"ahler MMP reading notes} \hfill #2 }
				\vspace{4mm}
				\hbox to 5.78in { {\Large \hfill #5  \hfill} }
				\vspace{2mm}
				\hbox to 5.78in { {\em #3 \hfill #4} }
			}
		}
	\end{center}
	\vspace*{4mm}
}

\newcommand{\lecture}[4]{\handout{#1}{#2}{#3}{#4}{Lecture #1 (draft version)}}
\usepackage{amsthm}

\theoremstyle{definition}
\newtheorem{theorem}{Theorem}
\newtheorem{corollary}[theorem]{Corollary}
\newtheorem{lemma}[theorem]{Lemma}
\newtheorem{observation}[theorem]{Observation}
\newtheorem{proposition}[theorem]{Proposition}
\newtheorem{definition}[theorem]{Definition}
\newtheorem{claim}[theorem]{Claim}
\newtheorem{remark}[theorem]{Remark}
\newtheorem{fact}[theorem]{Fact}
\newtheorem{assumption}[theorem]{Assumption}

% 1-inch margins, from fullpage.sty by H.Partl, Version 2, Dec. 15, 1988.
\topmargin 0pt
\advance \topmargin by -\headheight
\advance \topmargin by -\headsep
\textheight 8.9in
\oddsidemargin 0pt
\evensidemargin \oddsidemargin
\marginparwidth 0.5in
\textwidth 6.5in

\parindent 0in
\parskip 1.5ex
%\renewcommand{\baselinestretch}{1.25}

%\usepackage{biblatex}


\usepackage{fancyhdr}
\pagestyle{fancy}
\lhead{K\"ahler MMP reading seminars}
\rhead{\thepage}
%\cfoot{center of the footer!}
\renewcommand{\headrulewidth}{0.4pt}
\renewcommand{\footrulewidth}{0.4pt}

\usepackage{hyperref}

\hypersetup{
	colorlinks=true,
	linkcolor=Blue,
	filecolor=YellowOrange,
	citecolor = WildStrawberry,      
	urlcolor=cyan,
}


\begin{document}
	
	\lecture{4 --- 06, 06, 2024}{Spring 2024}{}{Yi Li}
	\section{Overview}
	The aim of this note is to introduce the contraction theorems in the K\"ahler minimal model program.
	
	\section{Fujiki's blowing down theoren and Grauert contraction theorem}
	
	
	\section{Koll\'ar-Mori's extension of contraction theorem}
	\begin{theorem}[{\cite[Theorem 11.4]{KM92}}]
		
	\end{theorem}
	\section{Das-Hacon's contraction theorem for generalized K\"ahler PLT pairs}
	In this section we will introduce the contraction theorem of Das and Hacon. 
	\begin{theorem}[{\cite[Theorem 5.8]{DH24}}]
		Let $(X, S+B + \boldsymbol\beta)$ be a generalized PLT pair with $\lfloor S+B\rfloor=S$ irreducible, such that the following condition holds
		\begin{enumerate}
			\item $S$ is a $\mathbb{Q}$-Cartier divisor,
			\item There exist a contraction morphism $f: S\to T$ such that $-S|_S$ is $f$-ample,
			\item the restriction of the canonical class $-(K_X+S+B+\boldsymbol{\beta})|_S$ is K\"ahler over $T$.
		\end{enumerate}
		Then we can find a (bimeromorphic) contraction morphism $F:X\to Y$, with $F|_S = f$.
	\end{theorem}
	\begin{remark}
		Let us briefly sketh the idea of the proof. Compared with the Fujiki blowing down theorem, we do not have Cartier condition on $S$ and we do not have the vanishing of higher direct image (of conormal sheaf) condition in the statement. 
		
		Since $S$ is $\mathbf{Q}$-Cartier, there exists a $r \in \mathbf{Z}$ such that $rS$ is Cartier. We first show that there exists on the infinitesimal thickending $rS$, with positivities preserved under thickening. Second we apply the Serre vanishing and change of index trick showing that the higher direct image of the conormal sheaf vanishs. Then apply the Fujiki blowing down theorem yield the result.
		
		The major difficulity of the proof lies in showing the obstruction of infinitesimal extension vanish. To do this, we need adjunction for the generalized PLT pair and Kawamata-Viehweg vanishing for the complex analytic space. 
	\end{remark}
	\begin{proof}
		
	\end{proof}
	
	%	\begin{center}
		%		\begin{tikzcd}
			%			&&& {H^0(\mathcal{X},R^2\pi_*\mathcal{O}_{\mathcal{X}})} \\
			%			{} & {H^1(\mathcal{X},\mathcal{O}_{\mathcal{X}}^*)} & {H^2(\mathcal{X},\mathbb{Z})} & {H^2(\mathcal{X},\mathcal{O}_{\mathcal{X}})} & {} \\
			%			{} & {H^1(X_s,\mathcal{O}_{X_s}^*)} & {H^2(X_s,\mathbb{Z})} & {H^2(X_s,\mathcal{O}_{X_s})} & {} \\
			%			&&& {R^2\pi_*\mathcal{O}_{\mathcal{X}}(s)}
			%			\arrow["\cong", from=1-4, to=2-4]
			%			\arrow[from=2-1, to=2-2]
			%			\arrow[from=2-2, to=2-3]
			%			\arrow[from=2-2, to=3-2]
			%			\arrow[from=2-3, to=2-4]
			%			\arrow[from=2-3, to=3-3]
			%			\arrow[from=2-4, to=2-5]
			%			\arrow[from=2-4, to=3-4]
			%			\arrow[from=3-1, to=3-2]
			%			\arrow[from=3-2, to=3-3]
			%			\arrow[from=3-3, to=3-4]
			%			\arrow[from=3-4, to=3-5]
			%			\arrow["\cong", from=3-4, to=4-4]
			%		\end{tikzcd}
		%	\end{center}
	
	%	\begin{figure}[H]
		%		\centering
		%		\includegraphics[width=0.85\linewidth]{"Moishezon morphism definition"}
		%		\caption{Comparison between different definitions for Moishezon morphism}
		%		\label{fig:moishezon-morphism-definition}
		%	\end{figure}
	\bibliographystyle{amsalpha}
	\bibliography{mybib.bib}
\end{document}
