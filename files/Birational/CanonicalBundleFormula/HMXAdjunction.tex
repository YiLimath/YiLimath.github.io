\documentclass[11pt]{article}
\usepackage[dvipsnames]{xcolor}
\usepackage{latexsym}
\usepackage{amsmath}
\usepackage{mathrsfs}
\usepackage{tikz-cd}
\usepackage{amssymb}
\usepackage{amsthm}
\usepackage{epsfig}
\usepackage{graphicx}
\usepackage{float}
\newcommand{\handout}[5]{
	\noindent
	\begin{center}
		\framebox{
			\vbox{
				\hbox to 5.78in { {\bf Hacon-Mckernan-Xu's canonical bundle formula} \hfill #2 }
				\vspace{4mm}
				\hbox to 5.78in { {\Large \hfill #5  \hfill} }
				\vspace{2mm}
				\hbox to 5.78in { {\em #3 \hfill #4} }
			}
		}
	\end{center}
	\vspace*{4mm}
}

\newcommand{\lecture}[4]{\handout{#1}{#2}{#3}{#4}{Note #1 (draft version)}}
\usepackage{amsthm}

\theoremstyle{definition}
\newtheorem{theorem}{Theorem}
\newtheorem{corollary}[theorem]{Corollary}
\newtheorem{lemma}[theorem]{Lemma}
\newtheorem{observation}[theorem]{Observation}
\newtheorem{proposition}[theorem]{Proposition}
\newtheorem{definition}[theorem]{Definition}
\newtheorem{proofidea}[theorem]{PROOF IDEA}
\newtheorem{claim}[theorem]{Claim}
\newtheorem{remark}[theorem]{Remark}
\newtheorem{fact}[theorem]{Fact}
\newtheorem{assumption}[theorem]{Assumption}

% 1-inch margins, from fullpage.sty by H.Partl, Version 2, Dec. 15, 1988.
\topmargin 0pt
\advance \topmargin by -\headheight
\advance \topmargin by -\headsep
\textheight 8.9in
\oddsidemargin 0pt
\evensidemargin \oddsidemargin
\marginparwidth 0.5in
\textwidth 6.5in

\parindent 0in
\parskip 1.5ex
%\renewcommand{\baselinestretch}{1.25}

%\usepackage{biblatex}


\usepackage{fancyhdr}
\pagestyle{fancy}
\lhead{Adjunction Theory Reading Notes}
\rhead{\thepage}
%\cfoot{center of the footer!}
\renewcommand{\headrulewidth}{0.4pt}
\renewcommand{\footrulewidth}{0.4pt}

\usepackage{hyperref}

\hypersetup{
	colorlinks=true,
	linkcolor=Blue,
	filecolor=YellowOrange,
	citecolor = WildStrawberry,      
	urlcolor=cyan,
}


\begin{document}
	
	\lecture{8 --- 06, 06, 2025}{Summer 2025}{}{Yi Li}
	The aim of this note is to prove the following Hacon-Mckernan-Xu's subadjunction theorem and show varies applications. The major references are \cite{BAB} and \cite{ACC}.
	
	
	
	\begin{theorem}[{\cite[Theorem 4.2]{ACC}}]~Assume the following condition holds:\\
		(1) Let $(X, B)$ be a projective klt pair; 
		
		(2) $G \subset X$ is a subvariety with normalisation $F$, (we want to do subadjunction on this $F$),
		
		(3) $X$ is $\mathbb{Q}$-factorial near the generic point of $G$,
		
		(4) $\Delta \geq 0$ is an $\mathbb{R}$-Cartier divisor on $X$; and $(X, B+\Delta)$ is lc near the generic point of $G$, (the LC condition allow us to find some DLT model near $G$),
		
		(5) there is a unique non-klt place of this pair whose centre is $G$. 
		
		(6) Let $d \in \mathbb{N}$, and let $\Phi$ be a subset of $[0,1]$ that contains 1, and $\operatorname{dim} X=d$ and $B \in \Phi$. 
		
		We then use MMP arguement (see more detail below) find some DLT model $\psi: Y \to X,$ such that there exists unique component $S$ with coefficient 1, and a contraction $h:S\to F$. We then apply the canonical bundle formula $$K_S+\Xi_S \sim_{\mathbb{R}}h^*(K_F+\Theta_F +P_F),$$where $\Xi_S$ is some axillary boundary divisor (which will be contruced below) and $\Theta_F$ is the discriminant divisor and $P_F$ is the moduli divisor in the canonical bundle formula. 
		
		Then Hacon-Mckernan-Xu subadjunction theorem says that:
		
		(a) (coefficient control on discriminant part) The coefficient of the discriminant divisor $\Theta_F$ belong to
		$
		\Psi:=\left\{a \mid 1-a \in \operatorname{LCT}_{d-1}(D(\Phi))\right\} \cup\{1\}
		$, thus only depends on $B$ not $\Delta$.
		
		(b) (positivity on moduli part) $P_F$ is pseudo-effective. 
	\end{theorem}
	
	\tableofcontents
	
	
	\section{Proof of Hacon-Mckernan-Xu's subadjunction}
	\subsection{Step 0. Construction of the DLT model}
	
	\subsection{Step 1. Construction of the axillary divisor $\Xi_S$}
	
	\subsection{Step 2. Proof of the coefficient of discriminant part $\Theta_F$ lies in the DCC set $\Psi$}
	
	\subsection{Step 3. Proof of the pseudo-effectiveness of the moduli part $P_F$}
	
	\section{Applications of Hacon-Mckernan-Xu's subadjunction}
	
	\bibliographystyle{amsalpha}
	\bibliography{mybib.bib}
	
\end{document}
