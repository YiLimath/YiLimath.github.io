\documentclass[11pt]{article}
\usepackage[dvipsnames]{xcolor}
\usepackage{latexsym}
\usepackage{amsmath}
\usepackage{mathrsfs}
\usepackage{tikz-cd}
\usepackage{amssymb}
\usepackage{amsthm}
\usepackage{epsfig}
\usepackage{graphicx}
\usepackage{float}
\newcommand{\handout}[5]{
	\noindent
	\begin{center}
		\framebox{
			\vbox{
				\hbox to 5.78in { {\bf Nonvanishing conjecture reading seminars} \hfill #2 }
				\vspace{4mm}
				\hbox to 5.78in { {\Large \hfill #5  \hfill} }
				\vspace{2mm}
				\hbox to 5.78in { {\em #3 \hfill #4} }
			}
		}
	\end{center}
	\vspace*{4mm}
}

\newcommand{\lecture}[4]{\handout{#1}{#2}{#3}{#4}{Lecture #1 (draft version)}}
\usepackage{amsthm}

\theoremstyle{definition}
\newtheorem{theorem}{Theorem}
\newtheorem{corollary}[theorem]{Corollary}
\newtheorem{lemma}[theorem]{Lemma}
\newtheorem{observation}[theorem]{Observation}
\newtheorem{proposition}[theorem]{Proposition}
\newtheorem{definition}[theorem]{Definition}
\newtheorem{claim}[theorem]{Claim}
\newtheorem{remark}[theorem]{Remark}
\newtheorem{fact}[theorem]{Fact}
\newtheorem{assumption}[theorem]{Assumption}

% 1-inch margins, from fullpage.sty by H.Partl, Version 2, Dec. 15, 1988.
\topmargin 0pt
\advance \topmargin by -\headheight
\advance \topmargin by -\headsep
\textheight 8.9in
\oddsidemargin 0pt
\evensidemargin \oddsidemargin
\marginparwidth 0.5in
\textwidth 6.5in

\parindent 0in
\parskip 1.5ex
%\renewcommand{\baselinestretch}{1.25}

%\usepackage{biblatex}


\usepackage{fancyhdr}
\pagestyle{fancy}
\lhead{Birational Geometry reading notes}
\rhead{\thepage}
%\cfoot{center of the footer!}
\renewcommand{\headrulewidth}{0.4pt}
\renewcommand{\footrulewidth}{0.4pt}

\usepackage{hyperref}

\hypersetup{
	colorlinks=true,
	linkcolor=Blue,
	filecolor=YellowOrange,
	citecolor = WildStrawberry,      
	urlcolor=cyan,
}


\begin{document}
	
	\lecture{4 --- 06, 06, 2024}{Spring 2024}{}{Yi Li}
	\tableofcontents
	
	\section{Overview}
	The aim of this note is to introduce the non-vanishing conjecture. We will prove the classical Shokurov non-vanishing theorem. We then try to prove the BCHM non-vanishing, which says that if the non-vanishing holds in dimension $(n-1)$ and special finiteness and existence of minimal conjecture hold in dimension $n$ then the non-vanishing conjecture also true in dimension $n$. Note that by the spriling induction, this will implies the non-vanishing theorem under BCHM conditions. After that we will introduce some recent progress on non-vanishing conjecture.
	\section{Shokurov non vanishing theorem}
	
	\section{BCHM version nonvanishing theorem}
	We will prove the following theorem in this section.
	\begin{theorem}\label{BCHMnonvanishing}
		Assume non-vanishing conjecture in dimension $n-1$, special finiteness in dimension $n$ and existence of minimal model in dimension $n$. Then the non-vanishing conjecture is also true in dimension $n$.
	\end{theorem}
	We will divide the proof into several steps.
	\subsection{Deformation of effectiveness}
	The result of this subsection allows us to reduce the problem to the absolute setting. 
	
	\begin{lemma}[Deformation of effectiveness]
		
	\end{lemma}
	\subsection{Lifting sections from NKLT center}
	Note that we assume that the non-vanishing in dimension (n-1), our strategy is to apply the technique of lifting section from NKLT center. To be more precise, we try to reduce to problem into the setting of the following lemma.
	\begin{lemma}
		
	\end{lemma}
	
	
	\subsection{Improve the pair when positive part of Zariski decomposition is non-zero}
	\begin{lemma}[Improve that pair when positive part of Zariski decomposition is non-zero]
		
	\end{lemma}
	\begin{lemma}[Local uniqueness of canonical models]
		
	\end{lemma}
	\subsection{Existence of minimal model after inserting an ample twist}
	
	\subsection{Proof of the BCHM non-vanishing}
	Now we can prove the BCHM non-vanishing theorem.
	\begin{proof}[Proof of Theorem \ref{BCHMnonvanishing}]
		
	\end{proof}
	
	%	\begin{figure}[H]
		%		\centering
		%		\includegraphics[width=0.85\linewidth]{"Moishezon morphism definition"}
		%		\caption{Comparison between different definitions for Moishezon morphism}
		%		\label{fig:moishezon-morphism-definition}
		%	\end{figure}
	
	\bibliographystyle{amsalpha}
	\bibliography{mybib.bib}
\end{document}
