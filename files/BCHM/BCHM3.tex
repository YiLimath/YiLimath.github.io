\documentclass[11pt]{article}
\usepackage{latexsym}
\usepackage{amsmath}
\usepackage{mathrsfs}
\usepackage{tikz-cd}
\usepackage{amssymb}
\usepackage{amsthm}
\usepackage{epsfig}
\usepackage{graphicx}
\usepackage{float}
\newcommand{\handout}[5]{
	\noindent
	\begin{center}
		\framebox{
			\vbox{
				\hbox to 5.78in { {\bf BCHM reading seminars} \hfill #2 }
				\vspace{4mm}
				\hbox to 5.78in { {\Large \hfill #5  \hfill} }
				\vspace{2mm}
				\hbox to 5.78in { {\em #3 \hfill #4} }
			}
		}
	\end{center}
	\vspace*{4mm}
}

\newcommand{\lecture}[4]{\handout{#1}{#2}{#3}{#4}{Lecture #1 (draft version)}}
\usepackage{amsthm}

\theoremstyle{definition}
\newtheorem{theorem}{Theorem}
\newtheorem{corollary}[theorem]{Corollary}
\newtheorem{lemma}[theorem]{Lemma}
\newtheorem{observation}[theorem]{Observation}
\newtheorem{proposition}[theorem]{Proposition}
\newtheorem{definition}[theorem]{Definition}
\newtheorem{claim}[theorem]{Claim}
\newtheorem{fact}[theorem]{Fact}
\newtheorem{assumption}[theorem]{Assumption}

% 1-inch margins, from fullpage.sty by H.Partl, Version 2, Dec. 15, 1988.
\topmargin 0pt
\advance \topmargin by -\headheight
\advance \topmargin by -\headsep
\textheight 8.9in
\oddsidemargin 0pt
\evensidemargin \oddsidemargin
\marginparwidth 0.5in
\textwidth 6.5in

\parindent 0in
\parskip 1.5ex
%\renewcommand{\baselinestretch}{1.25}

\usepackage[backend=bibtex,style=IEEE,sorting=nty]{biblatex}
\addbibresource{mybib.bib}

\begin{document}
	\section{Overview}
	We will continue our discussion on the paper BCHM. Last time we proved two fundamental consequences of BCHM, the first one was two minimal models are connected by finite flops, the second one showed the existence and finiteness theorem of canonical models of moduli space of stable curves with genus $g$ and $n$ marked points with general type log canonical divisor.
	
	This time we will show the following results:
	
	(1) Finite generation of the Cox ring associated to a sequence of divisors, as a consequence the (smooth) log Fano varieties are Mori dream space.
	
	(2) When the log canonical divisor is not pseudo-effective, the pair admits a Mori fibre space structure.
	
	(3) The crepant extraction theorem: there exist a crepant resolution that the set of exceptional divisors coinside with the pre given set of divisors.
	
	(4) If time permits we will also prove the Batyrev’s theorem (theorem 1.3.5 of the paper) that the cone of nef curves is a rational polyhedron and any co-extreme ray is generated by the pull back of class of curves contracting by some Mori fibre space.
	
	\section{log Fano varieties are Mori dream space, finite generation of Cox ring associated to a sequence of divisors}
	
	We first prove the toy case the smooth Fano variety 
	\section{}
	\printbibliography
\end{document}