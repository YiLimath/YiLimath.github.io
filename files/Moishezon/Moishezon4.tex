\documentclass[11pt]{article}
\usepackage{latexsym}
\usepackage{amsmath}
\usepackage{mathrsfs}
\usepackage{tikz-cd}
\usepackage{amssymb}
\usepackage{amsthm}
\usepackage{epsfig}
\usepackage{graphicx}
\usepackage{float}
\newcommand{\handout}[5]{
	\noindent
	\begin{center}
		\framebox{
			\vbox{
				\hbox to 5.78in { {\bf Moishezon morphism reading seminars} \hfill #2 }
				\vspace{4mm}
				\hbox to 5.78in { {\Large \hfill #5  \hfill} }
				\vspace{2mm}
				\hbox to 5.78in { {\em #3 \hfill #4} }
			}
		}
	\end{center}
	\vspace*{4mm}
}

\newcommand{\lecture}[4]{\handout{#1}{#2}{#3}{#4}{Lecture #1 (draft version)}}
\usepackage{amsthm}

\theoremstyle{definition}
\newtheorem{theorem}{Theorem}
\newtheorem{corollary}[theorem]{Corollary}
\newtheorem{lemma}[theorem]{Lemma}
\newtheorem{observation}[theorem]{Observation}
\newtheorem{proposition}[theorem]{Proposition}
\newtheorem{definition}[theorem]{Definition}
\newtheorem{claim}[theorem]{Claim}
\newtheorem{remark}[theorem]{Remark}
\newtheorem{fact}[theorem]{Fact}
\newtheorem{assumption}[theorem]{Assumption}

% 1-inch margins, from fullpage.sty by H.Partl, Version 2, Dec. 15, 1988.
\topmargin 0pt
\advance \topmargin by -\headheight
\advance \topmargin by -\headsep
\textheight 8.9in
\oddsidemargin 0pt
\evensidemargin \oddsidemargin
\marginparwidth 0.5in
\textwidth 6.5in

\parindent 0in
\parskip 1.5ex
%\renewcommand{\baselinestretch}{1.25}

\usepackage[backend=bibtex,style=IEEE,sorting=nty]{biblatex}
\addbibresource{mybib.bib}


\usepackage{fancyhdr}
\pagestyle{fancy}
\lhead{Moishezon morphism reading seminars}
\rhead{\thepage}
%\cfoot{center of the footer!}
\renewcommand{\headrulewidth}{0.4pt}
\renewcommand{\footrulewidth}{0.4pt}


\begin{document}
	
	\lecture{4 --- 06, 06, 2024}{Spring 2024}{}{Yi Li}
	\section{Overview}
	Today we will continue our discussion on the paper Moishezon morphism. Recall that last time we proved that a proper surjective morphism equipped with a relatively big line bundle is locally bimeromorphic to a projective morphism. We also proved that if the base space is Moishezon, then the total space is Moishezon if and only if the morphism is Moishezon. And finally the restriction of the generic surjective morphism on the exceptional set is a Moishezon morphism.
	
	This time we will show:
	(1) The fiber of the Moishezon morphism is Moishezon,
	(2) The alternation property of the very big locus, general type locus, and Moishezon locus, which is either nowhere dense or contains some dense open subset.
	
	\section{The fiber of the Moishezon morphism is again Moishezon}
	We will start today's discussion by using \cite{Moishezonmorphism}, Lemma 15. to prove the following theorem
	\begin{theorem}[The fiber of the Moishezon morphism is again Moishezon, see \cite{Moishezonmorphism}, Corollary 16]\label{Moishezon-morphism}
	The fibers of a proper, Moishezon morphism are Moishezon.
	\end{theorem}
	\begin{remark}
		Before proving the theorem, we make a remark that Moishezon morphism is stable under base change (which can be viewed as a generation of the theorem \ref{Moishezon-morphism}), see also \cite{Moishezonmorphism} Remark 17.
	\end{remark}
	
	\begin{proof}
	Let $g: X \rightarrow S$ be a proper, Moishezon morphism. It is bimeromorphic to a projective morphism $X^{\mathrm{p}} \rightarrow S$. We may assume $X^{\mathrm{p}}$ to be normal. Let $Y$ be the normalization of the closure of the graph of $X \dashrightarrow X^{\mathrm{p}}$.
	
	Fix now $s \in S$. Let $Z_s \subset X_s$ be an irreducible component and $W_s \subset Y_s$ an irreducible component that dominates $Z_s$. 
	
	By image of a Moishezon space is Moishezon if the morphism is surjective, it is enough to show that $W_s$ is Moishezon.
	
	If $\pi: Y \rightarrow X^{\mathrm{p}}$ is generically an isomorphism along $W_s$, then $W_s$ is bimeromorphic to an irreducible component of $X_s^{\mathrm{p}}$, hence Moishezon. Otherwise $W_s \subset \operatorname{Ex}(\pi)$. Now $\operatorname{Ex}(\pi) \rightarrow X^{\mathrm{p}}$ is Moishezon by (15) and $\operatorname{dim} \operatorname{Ex}(\pi)<$ $\operatorname{dim} Y=\operatorname{dim} X$. So $W_s$ is contained in a fiber of $\operatorname{Ex}(\pi) \rightarrow S$, hence Moishezon by induction on the dimension.
	\end{proof}
	
	
	\section{Moishezon locus, fiberwise Moishezon morphism is locally Moishezon under smoothness assumption}
	
	We begin this Section by defining very big locus, general type locus and Moishezon locus. 

	\printbibliography	
	
\end{document}
